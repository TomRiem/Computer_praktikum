\chapter{Numerics}

the Rayleigh quotient minimization problem on the sphere $\mathbb{S}^{n-1}$. For a symmetric matrix $A \in \mathbb{R}^{n \times n}$, the unit-norm eigenvector, $v \in \mathbb{R}^n$, corresponding to the smallest eigenvalue, defines the two global minima, $\pm v$, of the Rayleigh quotient  
\begin{equation}\label{RayleighQuotient}
    \begin{split}
        f \colon \; \mathbb{S}^{n-1} & \to \mathbb{R} \\
        x & \mapsto x^{\mathrm{T}} A x 
    \end{split}
\end{equation}   
with its gradient 
\begin{equation*}
    \operatorname{grad} f(x) = 2(Ax - x x^{\mathrm{T}} A x).
\end{equation*}

\begin{algorithm}[H]
    \caption{Inverse SR1 Method}\label{RiemannianSR1Method}
    \begin{algorithmic}[1]
        \State Continuously differentiable real-valued function $f$ on $\mathbb{R}^n$, bounded below; initial iterate $x_0 \in \mathbb{R}^n$; initial $\spd$ matrix $B^{\mathrm{SR1}}_0 \in \mathbb{R}^{n \times n}$; convergence tolerance $\varepsilon > 0$. Set $k = 0$.
        \While{$\lVert \nabla f(x_k) \rVert > \varepsilon$}
            \State Compute the search direction $d_k = - B^{\mathrm{SR1}}_k \nabla f(x_k)$.
            \State Determine a stepsize $\alpha_k > 0$ which satisfies \cref{RiemannianWolfeConditions1} and \cref{RiemannianWolfeConditions2.2}.
            \State Set $x_{k+1} = x_k + \alpha_k d_k$.
            \State Set $s_k = x_{k+1} - x_k$ and $y_k = \nabla f(x_{k+1}) - \nabla f(x_k)$.
            \State Compute $B^{\mathrm{SR1}}_{k+1} \in \mathbb{R}^{n \times n}$ by means of \cref{inverseSR1formula}. 
            \State Set $k = k+1$.
        \EndWhile
        \State \textbf{Return} $x_k$.
    \end{algorithmic}
\end{algorithm}

\begin{table}[H]\label{tab:Results}
    \resizebox{\textwidth}{!}{
        \begin{tabular}{l l l l l }
            \toprule
            Manifold & \multicolumn{2}{c}{$\mathbb{S}^{99}$} & \multicolumn{2}{c}{$\mathbb{S}^{299}$}   \\ 
            \midrule
            & Time & Iterations & Time & Iterations  \\ 
            \midrule
            BFGS & $0$ & $79$ & $0$ & $88$  \\ 
            \midrule
            Inverse BFGS & $79$ & $0$ & $0$ & $88$   \\
            \midrule
            SR1 & $0$ & $106$ & $0$ & $317$  \\ 
            \midrule
            Inverse SR1 & $0$ & $115$ & $0$ & $263$   \\
            \midrule
            Stable SR1 & $0$ & $0$ & $0$ & $0$  \\ 
            \midrule
            Inverse Stable SR1 & $0$ & $0$ & $0$ & $0$   \\
            \bottomrule
        \end{tabular}
    }
    \caption{Comparison of the quasi-Newton methods.}
\end{table}
$n = 100$ SR1: 3 runs failed, InverseSR1: 1 run failed
$n = 100$ SR1: 5 runs failed, InverseSR1: 5 runs failed
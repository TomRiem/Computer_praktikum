\chapter{Numerics}

In this section the implementation of a practical RBFGS method as a function in the programming language Julia (\url{https://julialang.org}) is presented and its numerical behavior is tested by suitable optimization problems. Julia is a high-performance, dynamic programming language, its features are well-suited for numerical analysis and computational science and it has a built-in package manager. The RBFGS method (and variants thereof) presented here is implemented in the package \lstinline!Manopt.jl! (available at \url{https://manoptjl.org}, \cite{Bergmann:2019}), which provides a framework for optimization on manifolds and follows the same philosophy as the \textsc{Matlab} version (available at \url{https://www.manopt.org/}, see also \cite{BoumalMishraAbsilSepulchre:2014}). \lstinline!Manopt.jl! provides an easy access to optimization methods on manifolds for Julia, including example data and visualization methods. This project is build upon \lstinline!ManifoldsBase.jl! (available at \url{https://github.com/JuliaManifolds/ManifoldsBase.jl}), a generic interface to implement manifolds. Certain functions are extended for specific manifolds from the package \lstinline!Manifolds.jl! (available at \url{https://juliamanifolds.github.io/Manifolds.jl}), which aims to provide a library of manifolds to be used within the project. The implemented manifolds are accompanied by their mathematical formulae. \\

There are various approaches for implementing a numerical method, all of them focus on different properties and aspects of a function that can be applied in practice. Qi named the following aspects in her dissertation \cite{Qi:2011}, which are important for the practical implementation of an RBFGS method:
\begin{enumerate}
    \item An efficient numerical representation for points $x$ on $\mathcal{M}$, tangent vectors $\xi_x \in \tangent{x}$ and the inner products $g_x(\xi_x, \eta_x)$ on $\tangent{x}$.
    \item An implementation of the chosen retraction $\retract{x} \colon \; \tangent{x} \to \mathcal{M}$.
    \item Efficient formulae for $f(\cdot)$ and $\operatorname{grad} f(\cdot)$.
    \item An implementation of the chosen vector transport $\vectorTransportSymbol$.
    \item A method for solving $\mathcal{H}^{RBFGS}_{k} [\eta_k] = -\operatorname{grad} f(x_k)$ or alternatively, a method for computing $\eta_k = -\mathcal{B}^{RBFGS}_{k} [\operatorname{grad} f(x_k)]$. 
\end{enumerate}
Methods for the first four points are provided by the packages \lstinline!Manopt.jl!, \lstinline!Manifolds.jl! and \lstinline!ManifoldsBase.jl!. The main focus of this section is therefore on a practical and efficient implementation of the update formula \cref{RiemannianInverseBFGSFormula} for the operator $\mathcal{B}^{RBFGS}_{k}$, so that the search direction $\eta_k$ can be calculated in each iteration by simply applying the operator $\mathcal{B}^{RBFGS}_{k}$ to the gradient $\operatorname{grad} f(x_k) \in \tangent{x_k}$ and thus the method runs stable and finds the minimum of the given minimization problem as quickly as possible. \\
It should be noted that only the implementation of \cref{RiemannianInverseBFGSFormula} is considered, as used in \cref{InverseGlobalRiemannianBFGS-Method}, \cref{InverseGlobalRiemannianBFGS-MethodLockingCondition} and \cref{CautiousRBFGSMethod}.

\section{Approach Of Representing The RBFGS Formula For Operators Via Matrices}
\label{Section5.1}

If we would consider $\mathcal{B}^{RBFGS}_{k}$ as a linear operator from $\tangent{x_k}$ to $\tangent{x_k}$, then his acting on a tangent vector would be understood as a function in the package \lstinline!Manopt.jl!, i.e. $\mathcal{B}^{RBFGS}_{k}$ would get a tangent vector $\xi_{x_k} \in \tangent{x_k}$ as an argument and return a tangent vector $\mathcal{B}^{RBFGS}_{k} [\xi_{x_k}] \in \tangent{x_k}$ as function-value. If we look at the update formula \cref{RiemannianInverseBFGSFormula} we see that in order to execute the update, functions would have to be multiplied by a scalar, added to each other and then saved again as a function. This methodology seems to us to be too elaborate, which is the reason why we follow the idea to express the operator $\mathcal{B}^{RBFGS}_{k}$ by a matrix with real-valued entries, which is updated in every iteration by means of \cref{RiemannianInverseBFGSFormula}. \\

In this section we start from the following setup, which results from the requirements of \cref{InverseGlobalRiemannianBFGS-Method}, \cref{InverseGlobalRiemannianBFGS-MethodLockingCondition} and \cref{CautiousRBFGSMethod}: We assume that we have an optimization problem given by a continuously differentiable real-valued function $f$ on a Riemannian manifold $(\mathcal{M}, g)$. The initial point is $x_0 \in \mathcal{M}$. We assume that the isometric vector transport, $\vectorTransportSymbol^S$, with associated retraction, $\retractionSymbol$, fulfill the locking condition, \cref{LockingCondition}. \\
First we create an orthonormal basis $(e_1, \cdots, e_n)$ in the tangent space $\tangent{x_0}$, i.e. 
\begin{equation*}
    g_{x_0}(e_i, e_j) = \delta_{ij} = \begin{cases} 1 &: i=j \\ 0 &: i \neq j \end{cases}.
\end{equation*}
It follows that the inner product of two tangent vectors in $\tangent{x_0}$ is equal to the Euclidean inner product of their coordinate representations with respect to this basis, i.e. for all $\xi_{x_0}, \eta_{x_0} \in \tangent{x_0}$ it holds $g_{x_0}(\xi_{x_0}, \eta_{x_0}) = \hat{\xi_{x_0}}^{\mathrm{T}} \hat{\eta_{x_0}}$ where $\hat{\xi_{x_0}}, \hat{\eta_{x_0}} \in \mathbb{R}^n$ with $\xi_{x_0} = (\hat{\xi_{x_0}})_1 e_1 + \cdots + (\hat{\xi_{x_0}})_n e_n$ (and the same relation holds for $\eta_{x_0}$ and $\hat{\eta_{x_0}}$). \\
As already mentioned, the operator $\mathcal{B}^{RBFGS}_k$ which in theory approximates the Hessian inverse is stored as a real matrix. As explained in \cref{Section3.3}, there exist coordinate representations of operators on tangent spaces with respect to a chosen basis. If we assume that the first operator $\mathcal{B}^{RBFGS}_0$ is the identity operator on $\tangent{x_0}$ (or a multiple of it), its coordinate representation is the identity matrix (multiplied by the corresponding factor) with respect to the orthonormal basis $(e_1, \cdots, e_n)$, i.e.
\begin{equation*}
    c \; \id_{\tangent{x_0}}[\cdot] = \mathcal{B}^{RBFGS}_0[\cdot] \simeq \hat{\mathcal{B}}^{RBFGS}_0[\cdot] = B^{RBFGS}_0 = c \; I_{n \times n} \in \mathbb{R}^{n \times n}.
\end{equation*}
This matrix is of course $\spd$ as long as $c > 0$. One could also take any other $\spd$ matrix as initial matrix, because it represents the coordinates of a positive definite self-adjoint operator and vice versa. For convenience, we assume that we are initializing the method with a multiple of the identity. \\
Instead of transporting the operator $\mathcal{B}^{RBFGS}_k$ in each iteration ($\mathcal{B}^{RBFGS}_k \mapsto \widetilde{\mathcal{B}}^{RBFGS}_k = \vectorTransportDir{x_k}{\alpha_k \eta_k}[S] \circ \mathcal{B}^{RBFGS}_k \circ {\vectorTransportDir{x_k}{\alpha_k \eta_k}[S]}^{-1}$) for updating it, we transport the orthonormal basis, through which $B^{RBFGS}_k$ will be the coordinate representation of the operator $\mathcal{B}^{RBFGS}_k$, into the tangent space of the next iterate $\tangent{x_{k+1}}$, i.e. $\tilde{e}_j = \vectorTransportDir{x_k}{\alpha_k \eta_k}[S] \circ \vectorTransportDir{x_{k-1}}{\alpha_{k-1} \eta_{k-1}}[S] \circ \cdots \circ \vectorTransportDir{x_0}{\alpha_0 \eta_0}[S] (e_j)$. In order to avoid further indexes, from now on we denote $(\tilde{e}_1, \cdots, \tilde{e}_n)$ the tangent vectors in the corresponding tangent space, originated from $(e_1, \cdots, e_n) \subset \tangent{x_0}$ with any number of applications of the vector transport $\vectorTransportSymbol^S$ to them. Since the vector transport $\vectorTransportSymbol^S$ is assumed to be isometric, the family $(\tilde{e}_1, \cdots, \tilde{e}_n)$ remains an orthonormal basis of the corresponding tangent space after any number of iterations. \\
The $k$-iteration now runs as follows: We calculate the gradient at $x_k$, i.e. $\operatorname{grad} f(x_k) \in \tangent{x_k}$. We determine its coordinates with respect to the orthonormal basis $\operatorname{grad} f(x_k) \simeq \widehat{\operatorname{grad} f(x_k)}\in \mathbb{R}^n$. Then we calculate the coordinates of the tangent vector $\eta_k$ as in the Euclidean case, i.e. $\hat{\eta_k} = - B^{RBFGS}_k \widehat{\operatorname{grad} f(x_k)} \in \mathbb{R}^n$. We construct the search direction from the vector $\hat{\eta_k}$ and the orthonormal basis $(\tilde{e}_1, \cdots, \tilde{e}_n)$:
\begin{equation*}
    \eta_k = (\hat{\eta_k})_1 \tilde{e}_1 + \cdots + (\hat{\eta_k})_n \tilde{e}_n.
\end{equation*}
To make sure that $\eta_k$ is a descent direction, we require that $B^{RBFGS}_k$ is a positive definite matrix, since $g_{x_k}(\operatorname{grad} f(x_k), \eta_k) = - \widehat{\operatorname{grad} f(x_k)}^{\mathrm{T}} B^{RBFGS}_k \widehat{\operatorname{grad} f(x_k)}$. \\
Let $\alpha_k > 0$ be a suitably chosen stepsize that meets the Wolfe conditions. The next steps as usual, i.e. $x_{k+1} = \retract{x_k}(\alpha_k \eta_k)$, $s_k = \vectorTransportDir{x_k}{\alpha_k \eta_k}(\alpha_k \eta_k)[S]$ and $y_k = \beta^{-1}_k \operatorname{grad} f(x_{k+1}) - \vectorTransportDir{x_k}{\alpha_k \eta_k}(\operatorname{grad} f(x_k))[S]$, where $\beta_k$ by means of \cref{LockingConditionParameter}. \\
Instead of using the tangent vectors $s_k, y_k \in \tangent{x_{k+1}}$ to update the operator $\mathcal{B}^{RBFGS}_k$ by means of \cref{RiemannianInverseBFGSFormula}, we use their coordinates $\hat{s_k}, \hat{y_k} \in \mathbb{R}^n$ to update the matrix $B^{RBFGS}_k$ by means of \cref{inverseBFGSformula}. For this we transport the orthonormal basis into the tangent space of the new iterate, i.e. $(\tilde{e}_1, \cdots, \tilde{e}_n) \subset \tangent{x_{k+1}}$ and determine the coordinates of $s_k$ and $y_k$ with respect to this basis. It is obvious that $\hat{s_k}$ and $\hat{y_k}$ fulfill \cref{CurvatureCondition}, because by choosing $\alpha_k > 0$ that satisfies \cref{RiemannianWolfeConditions2.1}, $s_k$ and $y_k$ fulfill \cref{RiemannianCurvatureCondition} and it holds
\begin{equation}\label{RiemannianCurvatureConditionCurvatureCondition}
    0 < g_{x_{k+1}}(s_k, y_k) = \hat{s}^{\mathrm{T}}_k \hat{y}_k.
\end{equation}
Now we come to the step where the previously mentioned obstacles can easily be avoided by working with matrices. At the end of each iteration the matrix $B^{RBFGS}_k$ is updated by means of \cref{inverseBFGSformula} and stored. The new matrix $B^{RBFGS}_{k+1}$ is also $\spd$, because $B^{RBFGS}_k$ inherits these characteristics to it since \cref{RiemannianCurvatureConditionCurvatureCondition} holds (see \cref{thmUlbrichUlbrich13.4}). \\

It must be shown that the descent directions $\eta_k$ generated using the operator $\mathcal{B}^{RBFGS}_k$ are the same as those generated from the (transported) orthonormal basis $(\tilde{e}_1, \cdots, \tilde{e}_n) \subset \tangent{x_k}$ and the real-valued valued vector obtained by multiplying the matrix $B^{RBFGS}_k$ with the coordinate representation of the gradient $\widehat{\operatorname{grad} f(x_k)}$ in each iteration. We show by induction that the coordinates of $\mathcal{B}^{RBFGS}_k [\operatorname{grad} f(x_k)]$ with respect to the orthonormal basis $(\tilde{e}_1, \cdots, \tilde{e}_n)$ are the same as the vector $B^{RBFGS}_k \widehat{\operatorname{grad} f(x_k)}$ in each iteration. \\

Therefore we first have to take a closer look at the linear operator $\mathcal{B}^{RBFGS}_k$ with respect to the orthonormal basis $(\tilde{e}_1, \cdots, \tilde{e}_n) \subset \tangent{x_k}$. The application of $\mathcal{B}^{RBFGS}_k$ to a tangent vector $\xi_{x_k} \in \tangent{x_k}$ can be written as
\begin{equation}\label{OperatorRepresentationTangentVectors}
    \mathcal{B}^{RBFGS}_k [\xi_{x_k}] = \sum^{n}_{i=1} {b^{k}_i}^{\flat} (\xi_{x_k}) \tilde{e}_i = \sum^{n}_{i=1} g_{x_k} (b^{k}_i, \xi_{x_k}) \tilde{e}_i \in \tangent{x_k},
\end{equation}
where $(b^{k}_1, \cdots, b^{k}_n) \subset \tangent{x_k}$ can been understood as a representation of the operator $\mathcal{B}^{RBFGS}_k$. By using the musical isomorphism $({b^{k}_1}^{\flat}, \cdots, {b^{k}_n}^{\flat}) \subset \cotangent{x_k}$ and since $\cotangent{x_k}$ is the space of the real-valued linear functions on $\tangent{x_k}$, we immediately see that \cref{OperatorRepresentationTangentVectors} is for all $\xi_{x_k} \in \tangent{x_k}$ a linear combination of the orthonormal basis $(\tilde{e}_1, \cdots, \tilde{e}_n)$ and therefore is $\mathcal{B}^{RBFGS}_k [\cdot] = \sum^{n}_{i=1} {b^{k}_i}^{\flat} (\cdot) \tilde{e}_i$ a linear operator from $\tangent{x_k}$ to $\tangent{x_k}$. \\
The coordinates of the resulting tangent vector $\mathcal{B}^{RBFGS}_k [\xi_{x_k}]$ with respect to the orthonormal basis $(\tilde{e}_1, \cdots, \tilde{e}_n)$ are:
\begin{equation*}
    \widehat{\mathcal{B}^{RBFGS}_k [\xi_{x_k}]} = (g_{x_k} (b^{k}_1, \xi_{x_k}), \cdots, g_{x_k} (b^{k}_n, \xi_{x_k}))^{\mathrm{T}} = (\hat{b^{k}_1}^{\mathrm{T}} \hat{\xi_{x_k}}, \cdots, \hat{b^{k}_n}^{\mathrm{T}} \hat{\xi_{x_k}})^{\mathrm{T}} \in \mathbb{R}^n,
\end{equation*}
where $\hat{b^{k}_i}$ and $\hat{\xi_{x_k}}$ are the coordinate representations of $\xi_{x_k}, b^{k}_i \in \tangent{x_k}$ with respect to the orthonormal basis $(\tilde{e}_1, \cdots, \tilde{e}_n)$. We require that the coordinate representation of the tangent vector $\mathcal{B}^{RBFGS}_k [\xi_{x_k}]$ is equal to the vector resulting from the coordinate representation of the operator, $\hat{\mathcal{B}}^{RBFGS}_k$, multiplied by the coordinate representation of the tangent vector, $\hat{\xi_{x_k}}$, i.e.
\begin{equation*}
    (\hat{b^{k}_1}^{\mathrm{T}} \hat{\xi_{x_k}}, \cdots, \hat{b^{k}_n}^{\mathrm{T}} \hat{\xi_{x_k}})^{\mathrm{T}} = \hat{\mathcal{B}}^{RBFGS}_k \hat{\xi_{x_k}}.
\end{equation*}
Thus we see that the lines of $\hat{\mathcal{B}}^{RBFGS}_k$ must be equal to the coordinate representations of $b^{k}_i$ for $i = 1, \cdots, n$, i.e. $(\hat{\mathcal{B}}^{RBFGS}_k)_{ij} = (\hat{b^{k}_i})_j = \hat{b^{k}}_{ij}$ where $\hat{b^{k}}_{ij}$ is the $j$-th entry of the vector $\hat{b^{k}}_i \in \mathbb{R}^n$. \\

Now we come to the actual induction. In the first iteration we represent the operator $\mathcal{B}^{RBFGS}_0[\cdot] = c \; \id[\cdot]$ by the basis tangent vectors multiplied by $c$, i.e. $(b^{0}_1, \cdots, b^{0}_n) = (c \; e_1, \cdots, c \; e_n)$, since $\operatorname{grad} f(x_0) = (\widehat{\operatorname{grad} f(x_0)})_1 e_1 + \cdots + (\widehat{\operatorname{grad} f(x_0)})_n e_n$, it holds
\begin{align*}
    \mathcal{B}^{RBFGS}_0[\operatorname{grad} f(x_0)] & = \sum^{n}_{i=1} g_{x_0} (b^{0}_i, \operatorname{grad} f(x_0)) e_i = \sum^{n}_{i=1} g_{x_0} (c \; e_i, \operatorname{grad} f(x_0)) e_i = \\
     & = c \; \sum^{n}_{i=1} (\widehat{\operatorname{grad} f(x_0)})_i e_i = c \operatorname{grad} f(x_0).
\end{align*}
The coordinate representation of $(c \; e_1, \cdots, c \; e_n)$ with respect to the orthonormal basis $(e_1, \cdots, e_n)$ are of course the canonical unit vectors multiplied by $c$ in $\mathbb{R}^n$. Therefore, since the coordinate representation of the tangent vectors $(b^{0}_1, \cdots, b^{0}_n)$ must correspond to the lines of the matrix $B^{RBFGS}_0$, the initial matrix is $B^{RBFGS}_0 = c \; I_{n \times n} \in \mathbb{R}^{n \times n}$. \\
We have to find an update formula for the tangent vectors $(b^{k}_1, \cdots, b^{k}_n)$ representing the operator $\mathcal{B}^{RBFGS}_k$ which is equal to \cref{RiemannianInverseBFGSFormula} and show that the coordinates of the resulting tangent vectors $(b^{k+1}_1, \cdots, b^{k+1}_n)$ in the new tangent space $\tangent{x_{k+1}}$ are the same as the line entries of the matrix $B^{RBFGS}_{k+1}$, which was generated by \cref{inverseBFGSformula}. Let $(\tilde{e}_1, \cdots, \tilde{e}_n) = (\vectorTransportDir{x_k}{\alpha_k \eta_k} (e_1), \cdots, \vectorTransportDir{x_k}{\alpha_k \eta_k} (e_n))$ be the orthonormal basis of $\tangent{x_{k+1}}$ and $s_k, y_k \in \tangent{x_{k+1}}$ such that $g_{x_{k+1}}(s_k, y_k) > 0$. It is required that 
\begin{align*}
    & \mathcal{B}^{RBFGS}_{k+1} [\xi_{x_{k+1}}] = \\ 
    = & \widetilde{\mathcal{B}}^{RBFGS}_k [\xi_{x_{k+1}}] - \frac{s_k y^{\flat}_k[\widetilde{\mathcal{B}}^{RBFGS}_k [\xi_{x_{k+1}}]]}{y^{\flat}_k [s_k]} - \frac{\widetilde{\mathcal{B}}^{RBFGS}_k [y_k] s^{\flat}_k [\xi_{x_{k+1}}]}{s^{\flat}_k [y_k]} + \frac{s_k y^{\flat}_k[\widetilde{\mathcal{B}}^{RBFGS}_k [y_k]]s^{\flat}_k [\xi_{x_{k+1}}]}{(y^{\flat}_k [s_k])^2} + \frac{s_k s^{\flat}_k [\xi_{x_{k+1}}]}{s^{\flat}_k [y_k]} = \\
     = &\sum^{n}_{i=1} g_{k+1} (b^{k+1}_i, \xi_{x_{k+1}}) \tilde{e}_i,
\end{align*}
where $\widetilde{\mathcal{B}}^{RBFGS}_k [\xi_{x_{k+1}}] = \sum^{n}_{i=1} g_{x_k} (\widetilde{b}^{k}_i, \xi_{x_k}) \tilde{e}_i$ with $\widetilde{b}^{k}_i = \vectorTransport{x_k}{x_{k+1}} (b^{k}_i) = \hat{b^{k}}_{i1} \vectorTransportDir{x_k}{\alpha_k \eta_k} (e_1) + \cdots + \hat{b^{k}}_{in} \vectorTransportDir{x_k}{\alpha_k \eta_k} (e_n)$. We see immediately that $\widetilde{b}^{k}_i \in \tangent{x_{k+1}}$ and $b^{k}_i \in \tangent{x_k}$ have the same coordinate representation $\hat{b^{k}_i} \in \mathbb{R}^n$ but in different tangent spaces, therefore $\widetilde{\mathcal{B}}^{RBFGS}_k \simeq B^{RBFGS}_k \in \mathbb{R}^{n \times n}$. After some calculations, the following formula can be used for the update of the tangent vectors:
\begin{equation}\label{InverseRBFGSUpdateRows}
    b^{k+1}_{i} = \widetilde{b}^{k}_i - \frac{(\hat{s_k})_i \; \widetilde{\mathcal{B}}^{RBFGS}_k [y_k]}{g_{x_{k+1}}(s_k, y_k)} - \frac{(\widehat{\widetilde{\mathcal{B}}^{RBFGS}_k [y_k]})_i \; s_k}{g_{x_{k+1}}(s_k, y_k)} + \frac{g_{x_{k+1}}(y_k, \widetilde{\mathcal{B}}^{RBFGS}_k [y_k]) \; (\hat{s_k})_i \; s_k}{(g_{x_{k+1}}(s_k, y_k))^2} + \frac{(\hat{s_k})_i \; s_k}{g_{x_{k+1}}(s_k, y_k)},
\end{equation}
where $(\hat{s_k})_i$ and $(\widehat{\widetilde{\mathcal{B}}^{RBFGS}_k [y_k]})_i$ is the $i$-th entry of the corresponding coordinate representation. \\
Next we look at the update of the matrix $B^{RBFGS}_k \mapsto B^{RBFGS}_{k+1}$ generated by \cref{inverseBFGSformula}, also line by line. Let $\hat{s_k}, \hat{y_k}$ be the coordinate representation of $s_k, y_k$ with respect to the orthonormal basis $(\tilde{e}_1, \cdots, \tilde{e}_n) \subset \tangent{x_{k+1}}$. With simple transformations it can be shown that the lines of the matrix are updated in each iteration as follows:
\begin{equation*}
    (B^{RBFGS}_{k+1})_{i:} = b^{k+1}_i = b^{k}_i - \frac{(\hat{s_k})_i \; B^{RBFGS}_k \hat{y}_k}{\hat{s}^{\mathrm{T}}_k \hat{y}_k} - \frac{(B^{RBFGS}_k \hat{y}_k)_i \; \hat{s}_k}{\hat{s}^{\mathrm{T}}_k \hat{y}_k} + \frac{\hat{y}^{\mathrm{T}}_k B^{RBFGS}_k \hat{y}_k \; (\hat{s_k})_i \; \hat{s}_k}{(\hat{s}^{\mathrm{T}}_k \hat{y}_k)^2} + \frac{(\hat{s_k})_i \; \hat{s}_k}{\hat{s}^{\mathrm{T}}_k \hat{y}_k},
\end{equation*}
where $b^{k}_i$ is the $i$-th line of the matrix $B^{RBFGS}_k$, $(\hat{s_k})_i$ and $(B^{RBFGS}_k \hat{y}_k)_i$ is the $i$-th entry of the corresponding vector. \\
We see immediately that this corresponds to the coordinate representation of \cref{InverseRBFGSUpdateRows} with respect to the basis $(\tilde{e}_1, \cdots, \tilde{e}_n) \subset \tangent{x_{k+1}}$, since $\hat{\widetilde{b}^{k}_i} = \hat{b^{k}_i} = b^{k}_i$, $g_{x_{k+1}}(s_k, y_k) = \hat{s}^{\mathrm{T}}_k \hat{y}_k$ and $\widehat{\widetilde{\mathcal{B}}^{RBFGS}_k [y_k]} = \hat{\widetilde{\mathcal{B}}}^{RBFGS}_k \hat{y_k} = B^{RBFGS}_k \hat{y_k}$ holds. \\
It follows that the coordinate representation of the operator $\mathcal{B}^{RBFGS}_k$, which is updated by means of \cref{RiemannianInverseBFGSFormula}, with respect to the transported basis $(\tilde{e}_1, \cdots, \tilde{e}_n)$ is equal to the matrix $B^{RBFGS}_k$, which is updated by means of \cref{inverseBFGSformula}, in each iteration. Thus the effect on the coordinates of $\operatorname{grad} f(x_k)$ caused by the application of the operator $\mathcal{B}^{RBFGS}_k$ with respect to the orthonormal basis is the same as that caused by the matrix-vector-multiplication $B^{RBFGS}_k \widehat{\operatorname{grad} f(x_k)}$ in each iteration. This means that a method in which the operator is expressed from the beginning by a matrix, which is updated and stored, is equivalent to the usual RBFGS method. \\

For the cautious RBFGS method, \cref{CautiousRBFGSMethod}, it should be noted that if the decision rule is not fulfilled, the matrix $B^{CRBFGS}_k$ is not updated, i.e. $B^{CRBFGS}_{k+1} = B^{CRBFGS}_k$, but the orthonormal basis $(\tilde{e}_1, \cdots, \tilde{e}_n)$ is still transported into the new tangent space $\tangent{x_{k+1}}$, because it is needed for the coordinate representation of the tangent vectors $\operatorname{grad} f(x_{k+1}), \eta_{k+1}, s_{k}, y_{k} \in \tangent{x_{k+1}}$ occurring there. \\
For the direct RBFGS method (in which the approximation $\mathcal{H}^{RBFGS}_k$ of the Hessian is used) this approach can also be followed, but then of course the update formula \cref{directBFGSformula} is used. One has to keep in mind that for an efficient method an efficient way to solve the linear system $\hat{\mathcal{H}}^{RBFGS}_k \hat{\eta}_k = -\widehat{\operatorname{grad} f(x_k)}$ is needed. 


\section{Experiments}

In this section the performance of the RBFGS method and the cautious LRBFGS method, implemented in \lstinline!Manopt.jl!, are investigated. \\
The methods are applied to Riemannian optimization problems which are defined in Julia. Essential for those are a manifold \lstinline!M!, created with constructors from the package \lstinline!Manifolds.jl!, a real-valued function \lstinline!F!, a gradient \lstinline!grad_F!, which assigns a tangent vector to each point on the manifold, and a starting point \lstinline!x!, which in most cases is randomly generated by \lstinline!random_point(M)!. \\
To apply the LRBFGS method with a memory size of $20$, \cref{quasi_NewtonCall} must be executed. That means, if \lstinline!quasi_Newton(M,F,grad_F,x)! is executed with the only necessary parameters that define the optimization problem, the choice goes to the LRBFGS method. 

\begin{lstlisting}[caption={Call of the \lstinline!quasi_Newton!-function from \lstinline!Manopt.jl! with the necessary input arguments.}, label={quasi_NewtonCall}]
    quasi_Newton(M, F, grad_F, x) 
\end{lstlisting}

The name \lstinline!quasi_Newton()! implies that of course other methods, belonging to the class of Riemannian quasi-Newton methods, can also be executed with it. This can be done by the optional input arguments, which can be defined after the necessary ones. If the LBFGS method with a different memory size should be executed, \lstinline!memory_size::Int! must be set equal to the number of vectors to be stored, starting from \lstinline!0!. The standard memory size is \lstinline!20!. If the LRBFGS method is not to be used, but a method where an operator is updated by means of \cref{RiemannianInverseBFGSFormula} (following the methodology from \cref{Section5.1}), \lstinline!memory_size! must be set equal to a negative number, e.g. \lstinline!memory_size = -1!. Since the update is realized by means of coordinates, an initial operator can be specified by means of \lstinline!initial_operator::AbstractMatrix! as a matrix. The default initial operator is \lstinline!Matrix(I,manifold_dimension(M), manifold_dimension(M))!, which represents the identity in the tangent space. By default the method is carried out with the exponential map, \lstinline!ExponentialRetraction()!, and the parallel transport, \lstinline!ParallelTransport()!. If this is not the case, this can be adjusted via \lstinline!retraction_method::AbstractRetractionMethod! and \lstinline!vector_transport_method::AbstractVectorTransportMethod!. \\ 
A cautious RBFGS and a cautious LRBFGS method can also be used. This can be done by setting \lstinline!cautious_update::Bool! to \lstinline!true!, the default is \lstinline!false!. For the cautious LRBFGS method, short CLRBFGS, the decision rule
\begin{equation}\label{CautiousTrigger}
    \frac{g_{x_{k+1}}(y_k,s_k)}{\lVert s_k \rVert^{2}_{x_{k+1}}} \geq \theta(\lVert \operatorname{grad} f(x_k) \rVert_{x_k})
\end{equation}
is used in each iteration to decide whether the vectors $s_k, y_k$ will be stored or not. The function $\theta$ can be set via \lstinline!cautious_function::Function!. Care must be taken that it is a monotone increasing function satisfying $\theta(0) = 0$ and it is strictly increasing at $0$. By default \lstinline!(x)->x*10^(-4)! is used. \\
As we found out in this thesis, the stepsize plays an essential role for the methods. The method to determine a stepsize can be set via \lstinline!step_size::Stepsize!. By default \lstinline!WolfePowellLineseach()!, which is a generalization of \cite[Algorithmus~9.3]{UlbrichUlbrich:2012}, that determines a stepsize which fulfills \cref{RiemannianWolfeConditions1} and \cref{RiemannianWolfeConditions2.2}, with the chosen \lstinline!retraction_method! and \lstinline!vector_transport_method! is used. The constants in the Wolfe conditions are set to $c_1 = 10^{−4}$ and $c_2 = 0.999$. We point out that the resulting stepsize does not lead to the fulfillment of the curvature condition \cref{RiemannianCurvatureCondition}, unless \lstinline!ExponentialRetraction()! and \lstinline!ParallelTransport()! or a combination of retraction and vector transport, that fulfills the locking condition \cref{LockingCondition}, are used. The criteria by which the method is aborted can be adjusted by \lstinline!stopping_criterion::StoppingCriterion!. By default, the method aborts either when $1000$ iterations have been reached or when the norm of the gradient is lesser than $10^{-6}$. \\
The numerical experiments are implemented in the toolbox \lstinline!Manopt.jl!. They were run on a Lenovo ThinkPad L490, 64 bit Windows system, 1.8 Ghz Intel Core i7-8565U, 32 GB RAM, with Julia 1.5.2.


\subsection{Rayleigh Quotient Minimization}
\label{Section5.2.1}

To show the performance of the RBFGS method, implemented in \lstinline!Manopt.jl!, which follows the concept from \cref{Section5.1}, we compare the average time per run and the average number of iterations with the results from \cite[p.~84]{Qi:2011} with respect to the Rayleigh quotient minimization problem on the sphere $\mathbb{S}^{n-1}$. For a symmetric matrix $A \in \mathbb{R}^{n \times n}$, the unit-norm eigenvector, $v \in \mathbb{R}^n$, corresponding to the smallest eigenvalue, defines the two global minima, $\pm v$, of the Rayleigh quotient  
\begin{equation}\label{RayleighQuotient}
    \begin{split}
        f \colon \; \mathbb{S}^{n-1} & \to \mathbb{R} \\
        x & \mapsto x^{\mathrm{T}} A x 
    \end{split}
\end{equation}   
with its gradient 
\begin{equation*}
    \operatorname{grad} f(x) = 2(Ax - x x^{\mathrm{T}} A x).
\end{equation*}
To apply the RBFGS method, implemented in \lstinline!Manopt.jl!, to the optimization problem defined by the cost function \cref{RayleighQuotient} for $n=100$, \cref{RayleighCode} must be executed in Julia. The problem is defined by setting \lstinline!A_symm = ( A + A' ) / 2!, where the elements of \lstinline!A! are drawn from the standard normal distribution using Julia’s \lstinline!randn(n,n)! with seed \lstinline!42!. With \lstinline!random_point(M)!, a random point is created on the given manifold \lstinline!M!. The stopping criterion requires to abort the method, that the ratio of the norm of the initial gradient and the norm of the current gradient is less than $10^{-6}$.\\ 

\begin{lstlisting}[caption={The Rayleigh quotient minimization experiment in Julia for $n = 100$.}, label={RayleighCode}]
    using Manopt, Manifolds, Random
    Random.seed!(42)
    n = 100
    A = randn(n,n)
    A_symm = ( A + A' ) / 2
    M = Sphere(n - 1)
    F(X::Array{Float64,1}) = X' * A_symm * X
    grad_F(X::Array{Float64,1}) = 2 * ( A_symm * X - X * X' * A_symm * X )
    x = random_point(M)
    
    quasi_Newton(M, 
        F, 
        grad_F, 
        x; 
        memory_size = -1, 
        stopping_criterion = StopWhenGradientNormLess(norm(M, x, grad_F(x)) * 10^(-6))) 
\end{lstlisting}

\begin{table}[H]\label{tab:RayleigResults}
    \resizebox{\textwidth}{!}{
        \begin{tabular}{l l l l l l l }
            \toprule
            Manifold & \multicolumn{3}{c}{$\mathbb{S}^{99}$}& \multicolumn{3}{c}{$\mathbb{S}^{299}$} \\ 
            \midrule
            Method & RBFGS & RBFGS-Qi-1 & RBFGS-Qi-2 & RBFGS & RBFGS-Qi-1 & RBFGS-Qi-2  \\ 
            \midrule
            Time in seconds & $0.15$ & $0.21$ & $0.54$ & $0.96$ & $4.6$ & $11.0$ \\ 
            \midrule
            Iterations & $72$ & $68$ & $72$ & $70$ & $92$ & $97$ \\
            \bottomrule
        \end{tabular}
    }
    \caption{Comparison of RBFGS from \lstinline!Manopt.jl! with RBFGS from \cite{Qi:2011} on $\mathbb{S}^{n-1}$ for $n=100,300$.}
\end{table}


\cref{tab:RayleigResults} contains the results of the RBFGS method from \lstinline!Manopt.jl! and the RBFGS method from \cite{Qi:2011} on $\mathbb{S}^{99}$ and $\mathbb{S}^{299}$. The RBFGS columns represent the results obtained by \lstinline!quasi_Newton()!. \cref{RayleighCode} was used in the experiments depending on the parameter \lstinline!n!. For the time measurement the package BenchmarkTools.jl was used. The time given is the average of $10$ random runs, where the time in each run was measured with a benchmark of $10$ samples and $1$ evaluation per sample. The iterations are the average of the measured iterations from $10$ random runs. In columns RBFGS-Qi-1 and RBFGS-Qi-2, the results from \cite[Table~5.1]{Qi:2011} have been included. \\
For the manifold $\mathbb{S}^{99}$, the values generated by executing the different methods do not differ significantly. But for $\mathbb{S}^{299}$ the RBFGS method from \lstinline!Manopt.jl! seems to achieve better times and lesser iterations. The big time difference can be explained by the fact that both methods from \cite{Qi:2011} seem to be implemented and tested in \textsc{Matlab}, since reference is made to the \lstinline!GenRTR! package (Generic Riemannian Trust-Region, available at \url{https://www.math.fsu.edu/~cbaker/GenRTR/}) and therefore time differences are not surprisingly. \\
More interesting are the average iterations, which are also less for the RBFGS method from \lstinline!Manopt.jl! than for the RBFGS methods from \cite{Qi:2011}. This can be explained by the fact that all three methods have a different approach, how the operators $\mathcal{B}^{RBFGS}_k$ are realized and how the search direction $\eta_k$ results from it. \\
The method generating the results for RBFGS-Qi-1 follows the approach that $\mathcal{M}$ is considered as a submanifold of $\mathbb{R}^{m}$ and its tangent spaces $\tangent{x}$ are identified with subspaces of $\mathbb{R}^{m}$. It realizes the operator $\mathcal{B}^{RBFGS-Qi-1}_k$ as a matrix $B^{(m)}_k \in \mathbb{R}^{m \times m}$ such that $B^{(m)}_k i_{x_k}(\xi_{x_k}) = i_{x_k}(\mathcal{B}^{RBFGS-Qi-1}_k[\xi_{x_k}])$ holds, where $i_x \colon \; \tangent{x} \to \mathbb{R}^{m}, \; \xi_x \mapsto i_x(\xi_x)$ denotes the natural inclusion of $\tangent{x}$ in $\mathbb{R}^{m}$. The descent direction is given by $\eta_k = i^{-1}_{x_k}(-B^{(m)}_k i_{x_k}(\operatorname{grad} f(x_k)))$. Of course this matrix $B^{(m)}_k$ is also updated by means of \cref{inverseBFGSformula} using $i_{x_{k+1}}(s_k), i_{x_{k+1}}(y_k) \in \mathbb{R}^m$ and $\widetilde{B}^{(m)}_k = T^{(m)}_{\alpha_k \eta_k} B^{(m)}_k (T^{(m)}_{\alpha_k \eta_k})^{\dagger}$ where $T^{(m)}_{\alpha_k \eta_k} \in \mathbb{R}^{m \times m}$ is the matrix representation of the chosen vector transport $\vectorTransportDir{x_k}{\alpha_k \eta_k}$ satisfying $T^{(m)}_{\alpha_k \eta_k} i_{x_k}(\xi_{x_k}) = i_{x_k}(\vectorTransportDir{x_k}{\alpha_k \eta_k}(\xi_{x_k}))$ and $\dagger$ denotes the pseudoinvers. For this approach, the typical situation is that $T^{(m)}_{\alpha_k \eta_k}$ is expressed efficiently via some projection-like expression (see \cite[p.~44-46]{Qi:2011}) and $B^{(m)}_k, \widetilde{B}^{(m)}_k, B^{(m)}_{k+1}$ will be expressed as dense $m \times m$ matrices that are singular but $\spd$ on the subspaces representing the appropriate tangent space of $\mathcal{M}$ (see \cite[3.3.1~Approach~1]{Qi:2011}). \\
The RBFGS method from \lstinline!Manopt.jl! and the method generating the results for RBFGS-Qi-2 methods are similar. RBFGS-Qi-2 realizes the operator $\mathcal{B}^{RBFGS-Qi-2}_k$ as a matrix $B^{(n)}_k \in \mathbb{R}^{n \times n}$ and a basis $(e_1, \cdots, e_n)$ of $\tangent{x_k}$, where $n$ is the dimension of the manifold $\mathcal{M}$. The coordinates of the search direction $\eta_k$ are given by $\hat{\eta_k} = - B^{(n)}_k \widehat{\operatorname{grad} f(x_k)} \in \mathbb{R}^n$ and the resulting search direction is $\eta_k = \sum^{n}_i (\hat{\eta_k})_i e_i \in \tangent{x_k}$. At the end, the matrix $B^{(n)}_k$ is updated by means of \cref{inverseBFGSformula} (see \cite[3.3.2~Approach~2]{Qi:2011}). \cite{Qi:2011} states that the main efficiency concern of this approach is the cost of transport and/or creation of the required bases and the cost of obtaining the coordinates. In general, the RBFGS-Qi-1 is also not completely independent of tangent space bases computationally, because of its vector transport. Since RBFGS-Qi-2 works with the coordinates with respect to a basis, it uses the basis not only for its vector transport but also for updating the matrix $B^{(n)}_k$. As a result, it is expected to be computationally more costly. Which of these methods is superior depends on the manifold type and the size of the manifold which is critical in the complexity of creating basis of the tangent space. \cite{Qi:2011} further states that RBFGS-Qi-2 has potential, as the dimension of the manifold is in general smaller than the space in which it is embedded, but only if there is an efficient manner of obtaining the coordinates with respect to the basis. \\
Exactly this last aspect seems to have been well realized in the RBFGS method from \lstinline!Manopt.jl! by using the functions \lstinline!get_coordinates(M,x,xi,basis)!, which returns the coordinates of a tangent vector \lstinline!xi! in the tangent space of \lstinline!x! with respect to a \lstinline!basis!, and \lstinline!get_vector(M,x,coordinates,basis)!, which returns a tangent vector of the tangent space of \lstinline!x! formed from the \lstinline!coordinates! and the \lstinline!basis!, from the package \lstinline!ManifoldsBase.jl!. This could be the decisive reason for the fewer iterations, as it can be assumed that its implementation makes the computation of the descent directions “more accurate” in each iteration and thus the algorithm reaches a stationary point faster. 
\subsection{Brockett Cost Function}
\label{Section5.2.2}

To show the performance of the cautious LRBFGS method, implemented in \lstinline!Manopt.jl!, we compare the average time per run and the average number of iterations with the results from \cite[p.~1683]{HuangGallivanAbsil:2015} with respect to the Brockett cost function minimization problem on the Stiefel manifold $St(k,n)$. For a symmetric matrix $A \in \mathbb{R}^{n \times n}$, the eigenvectors for the $k$ smallest eigenvalues $\lambda_1 \leq \cdots \leq \lambda_k$ are the columns of a global minimizer of
\begin{equation}\label{BrockettCostFunction}
    \begin{split}
        f \colon \; St(k,n) & \to \mathbb{R} \\
        X & \mapsto \operatorname{trace}(X^{\mathrm{T}} A X N) 
    \end{split}
\end{equation}
where $N = diag(\mu_1, \cdots, \mu_p)$ with $\mu_1 > \cdots > \mu_p > 0$. \cref{BrockettCostFunction} is not a retraction-convex function on the entire domain. However, for generic choices of $A$, the function is strongly retraction-convex in a sublevel set around any global minimizer \cite[p.~1678]{HuangGallivanAbsil:2015}. \cref{BrockettCostFunction} can be considered as a weighted sum $\sum_i \mu_i x^{\mathrm{T}}_i A x_i$ of Rayleigh quotients on the sphere under an orthogonality constraint $x^{\mathrm{T}}_i x_j = \delta_{ij}$. Its gradient is given by 
\begin{equation*}
    \operatorname{grad} f(X) = 2 A X N - X X^{\mathrm{T}} A X N - X N X^{\mathrm{T}}.
\end{equation*}

Unfortunately, we are unaware of any closed expression for the parallel transport on the Stiefel manifold \cite{EdelmanAriasSmith:1998}. Therefore we cannot use the default choice of exponential map, \lstinline!ExponentialRetraction()!, and parallel transport, \lstinline!ParallelTransport()!, for solving this optimization problem. A closed form expression of the exponential map is known for the Stiefel manifold, but it is computationally very expensive. \\
We decided to conduct this experiment with the QR-based retraction on the Stiefel manifold. With $Q R = x_k + \alpha_k \eta_k \in \mathbb{R}^{n \times k}$ the retraction reads
\begin{equation}\label{QRRetraction}
    \retract{x_k}(\alpha_k \eta_k) = Q D,
\end{equation}
where $D$ is a $n \times k$ matrix with $D = \operatorname{diag}({\operatorname{sgn}(R_{ii} + 0.5)}^k_{i = 1})$ and $\operatorname{sgn} \colon \; \mathbb{R} \to \{-1, 0, 1 \}$ is the well-known signum function. \\
We use for this experiment a projection-based vector transport, i.e.
\begin{equation}\label{VectorTransportProjection}
    \vectorTransport{x_k}{\retract{x_k}(\alpha_k \eta_k)}(\alpha_k \eta_k) = \proj{\retract{x_k}(\alpha_k \eta_k)} (\alpha_k \eta_k) 
\end{equation}
where $\proj_x \colon \; \mathbb{R}^{n \times k} \to \tangent{x}, \; \proj{x}(M) = M - x^{\mathrm{T}} \operatorname{Symm} (x^{\mathrm{T}}M)$ with $\operatorname{Symm}(A) = (A + A^\mathrm{T})/2$ projects a matrix $M$ onto the tangent space of $x$. We see that \cref{VectorTransportProjection} fulfills all the properties of \cref{VectorTransport}. We note that this vector transport with associated retraction (defined in \cref{QRRetraction}) is not isometric and is also not the vector transport by differentiated retraction of the chosen QR-based retraction. This implies of course that \cref{VectorTransportProjection} and \cref{QRRetraction} do not meet the locking condition, \cref{LockingCondition}. We take the liberty of this lack of precision in the experiments, as on the one hand no suitable combination of vector transport and retraction, which would fulfill the locking condition, is implemented in \lstinline!Manopt.jl! and on the other hand both \cref{VectorTransportProjection} and \cref{QRRetraction} offer implementations which are computationally cheap. \\
We set in each iteration
\begin{equation*}
    \begin{split}
        & s_k = \proj{\retract{x_k}(\alpha_k \eta_k)} (\alpha_k \eta_k) \\
        & y_k = \beta^{-1}_k \operatorname{grad} f(x_{k+1}) - \proj{\retract{x_k}(\alpha_k \eta_k)}(\operatorname{grad} f(x_k)) \\
        & \text{where} \quad \beta_k = \frac{\lVert \alpha_k \eta_k \rVert_{x_k}}{\lVert \proj{\retract{x_k}(\alpha_k \eta_k)} (\alpha_k \eta_k) \rVert_{x_{k+1}}}.
    \end{split}
\end{equation*}
We note that the operators $\mathcal{B}^{CLRBFGS}_k$ do not have to be positive definite, firstly because the vector transport, \cref{VectorTransportProjection}, is not isometric and secondly because the curvature condition, \cref{RiemannianCurvatureCondition}, is not always satisfied. This means that $\eta_k$ does not necessarily have to be a descent direction. We hope that by applying the decision rule, \cref{CautiousTrigger}, in the cautious update, we will still get a convergent method, because only vectors $s_k, y_k \in \tangent{x_{k+1}}$ which already fulfill the curvature condition, \cref{RiemannianCurvatureCondition}, will be stored. \\
To apply the CLRBFGS method with a memory size of $4$, implemented in \lstinline!Manopt.jl!, to the optimization problem defined by the cost function \cref{BrockettCostFunction} for $St(32,32)$, \cref{BrockettCode} must be executed in Julia. We point out that when the constructor to create a Stiefel manifold is executed, the dimensions are reversed, that means \lstinline!Stiefel(n,k)! creates the manifold $St(k,n)$ in Julia. The problem is defined by setting \lstinline!A_symm = A + A'!, where the elements of \lstinline!A! are drawn from the standard normal distribution using Julia’s \lstinline!randn(n,n)! with seed \lstinline!42! and \lstinline!N! is a diagonal matrix whose diagonal elements are integers from $k$ to $1$. With \lstinline!random_point(M)! a random point is created on the given manifold. The stopping criterion requires to abort the method, that the ratio of the norm of the initial gradient and the norm of the current gradient is less than $10^{-6}$. \\ 

\begin{lstlisting}[caption={The Brockett cost function experiment in Julia for $n = 32, k = 32$ and a memory size of $4$.}, label={BrockettCode}]
    using Manopt, Manifolds, LinearAlgebra, Random
    Random.seed!(42)
    n = 32
    k = 32
    M = Stiefel(n,k)
    A = randn(n,n)
    A_symm = A + A' 
    N = diagm(k : -1 : 1)
    F(X::Array{Float64,2}) = tr(X' * A_symm * X * N)
    grad_F(X::Array{Float64,2}) = 2 * A_symm * X * N - X * X' * A_symm * X * N - X * N * X' * A_symm * X
    x = random_point(M)
    
    quasi_Newton(M, 
        F, 
        grad_F, 
        x; 
        memory_size = 4, 
        cautious_update = true,
        retraction_method = QRRetraction(), 
        vector_transport_method = ProjectionTransport(), 
        stopping_criterion = StopWhenGradientNormLess(norm(M, x, grad_F(x)) * 10^(-6))) 
\end{lstlisting}


\begin{table}[H]\label{tab:BrockettResultsMemory1}
    \resizebox{\textwidth}{!}{
        \begin{tabular}{l l l l l l l}
            \toprule
            Memory size & \multicolumn{2}{c}{$1$} & \multicolumn{2}{c}{$2$} & \multicolumn{2}{c}{$4$}  \\ 
            \midrule
            Method & CLRBFGS & LRBFGS-HGA & CLRBFGS & LRBFGS-HGA & CLRBFGS & LRBFGS-HGA \\ 
            \midrule
            Time in seconds & $1.504$ & $0.653$ & $1.665$ & $0.662$ & $1.299$ & $0.741$ \\ 
            \midrule
            Iterations & $664$ & $760$ & $568$ & $678$ & $482$ & $609$  \\
            \bottomrule
        \end{tabular}
    }
    \caption{Comparison of CLRBFGS from \lstinline!Manopt.jl! with LRBFGS from \cite{HuangGallivanAbsil:2015} for a memory size of $1, 2, 4$.}
\end{table}


\begin{table}[H]\label{tab:BrockettResultsMemory2}
    \resizebox{\textwidth}{!}{
        \begin{tabular}{l l l l l l l}
            \toprule
            Memory size & \multicolumn{2}{c}{$8$} & \multicolumn{2}{c}{$16$} & \multicolumn{2}{c}{$32$}  \\ 
            \midrule
            Method & CLRBFGS & LRBFGS-HGA & CLRBFGS & LRBFGS-HGA & CLRBFGS & LRBFGS-HGA \\ 
            \midrule
            Time in seconds & $1.470$ & $0.973$ & $1.667$ & $1.36$ & $2.676$ & $2.08$ \\ 
            \midrule
            Iterations & $464$ & $584$ & $463$ & $538$ & $414$ & $491$  \\
            \bottomrule
        \end{tabular}
    }
    \caption{Comparison of CLRBFGS from \lstinline!Manopt.jl! with LRBFGS from \cite{HuangGallivanAbsil:2015} for a memory size of $8, 16, 32$.}
\end{table}

\cref{tab:BrockettResultsMemory1} and \cref{tab:BrockettResultsMemory2} contain the results of the cautious LRBFGS method from \lstinline!Manopt.jl! and the LRBFGS method from \cite{HuangGallivanAbsil:2015} for the memory sizes \lstinline!1,2,4,8,16,32!. The CLRBFGS columns represent the results obtained by \lstinline!quasi_Newton()!. \cref{BrockettCode} was used in the experiments depending on the input argument \lstinline!memory_size!. For the time measurement the package BenchmarkTools.jl was used. The time given is the average of $10$ random runs, where the time in each run was measured with a benchmark of $1$ sample and $1$ evaluation per sample. The iterations are the average of the measured iterations from $10$ random runs. In the LRBFGS-HGA columns, the results from \cite[Table~4]{HuangGallivanAbsil:2015} have been included. \\
The first thing to notice in \cref{tab:BrockettResultsMemory1} and \cref{tab:BrockettResultsMemory2} is the time difference between the two methods. The CLRBFGS method from \lstinline!Manopt.jl! is on average slower by a factor of $\sim 1.6$ per iteration. This can have several reasons. Of course it could again be attributed to the use of the different software. The experiments in \cite{HuangGallivanAbsil:2015} were also implemented and performed in \textsc{Matlab}. But it would be surprising if this is the reason, because in general Julia is faster and this experiment would be an exception. It could also simply be the hardware. The processor of the computer with which the experiments were performed in \cite{HuangGallivanAbsil:2015} reaches 3.6 Ghz. The most likely reason, however, is the different computation of the stepsize. In \cite{HuangGallivanAbsil:2015}, a generalization of \cite[Algorithm~A6.3.1mod]{DennisSchnabel:1996} is used, which determines a stepsize that fulfills the Wolfe conditions \cref{RiemannianWolfeConditions1} and \cref{RiemannianWolfeConditions2.1} with the same constants (i.e. $c_1 = 10^{−4}$ and $c_2 = 0.999$), and the initial stepsize is determined by a generalization of the approach in \cite[p.~60]{NocedalWright:2006}. This approach seems to be very sophisticated and could produce an appropriate stepsize quickly with little effort. This suggests that a generalization of \cite[Algorithmus~9.3]{UlbrichUlbrich:2012}, as we use it in \lstinline!quasi_Newton()!, may not be the first choice. \\
The average numbers of iterations in \cref{tab:BrockettResultsMemory1} and \cref{tab:BrockettResultsMemory2} are surprisingly lower for the CLRBFGS method from \lstinline!Manopt.jl!. It can therefore be assumed that the decision rule \cref{CautiousTrigger} is fulfilled in most cases. This means on the one hand that the curvature condition \cref{RiemannianCurvatureCondition} is fulfilled in most cases, which is surprising, since \cref{VectorTransportProjection} and \cref{QRRetraction} do not meet the locking condition, \cref{LockingCondition}, and on the other hand, since new information about the curvature is added to the memory by fulfilling the decision rule (through storing the vectors $s_k$ and $y_k$), that this information is well processed by the CLRBFGS method from \lstinline!Manopt.jl!, which suggests an efficient computation for $\eta_k$. \\

\begin{table}[H]\label{tab:BrockettResultsLargeDimensions}
    \resizebox{\textwidth}{!}{
        \begin{tabular}{l l l l l l l l l}
            \toprule
            Manifold & \multicolumn{2}{c}{$St(2,1000)$} & \multicolumn{2}{c}{$St(3,1000)$} & \multicolumn{2}{c}{$St(4,1000)$} & \multicolumn{2}{c}{$St(5,1000)$}  \\ 
            \midrule
            Method & CLRBFGS & LRBFGS-HGA & CLRBFGS & LRBFGS-HGA & CLRBFGS & LRBFGS-HGA & CLRBFGS & LRBFGS-HGA \\ 
            \midrule
            Time in seconds & $168.038$ & $0.807$ & $211.895$ & $1.70$ & $306.802$ & $2.70$ & $ 272.467$ & $4.48$ \\ 
            \midrule
            Iterations & $283$ & $233$ & $417$ & $368$ & $505$ & $449$ & $551$ & $526$ \\ 
            \bottomrule
        \end{tabular}
    }
    \caption{Comparison of CLRBFGS from \lstinline!Manopt.jl! with LRBFGS from \cite{HuangGallivanAbsil:2015} on $St(k,1000)$ for $k = 2, 3, 4, 5$.}
\end{table}

\cref{tab:BrockettResultsLargeDimensions} contains the results of the cautious LRBFGS method from \lstinline!Manopt.jl! and the LRBFGS method from \cite{HuangGallivanAbsil:2015} on $St(k,1000)$ for $k = 2, 3, 4, 5$. The CLRBFGS columns represent the results obtained by \lstinline!quasi_Newton()!. \cref{BrockettCode} was used in the experiments depending on the parameter \lstinline!k!. For the time measurement the package BenchmarkTools.jl was used. The time given is the average of $5$ random runs, where the time in each run was measured with a benchmark of $1$ sample and $1$ evaluation per sample. The iterations are the average of the measured iterations from $5$ random runs. In the LRBFGS-HGA columns, the results from \cite[Table~4]{HuangGallivanAbsil:2015} have been included. \\
The values in \cref{tab:BrockettResultsLargeDimensions} show that the performance of CLRBFGS method from \lstinline!Manopt.jl! rather cannot compete with the performance of the LRBFGS method from \cite{HuangGallivanAbsil:2015}. For the time differences, of course, the above arguments can be applied again, but also others associated with the higher average number of iterations. Most obviously, this could be attributed to the characteristic of $\eta_k$ not necessarily to be a descent direction, as already mentioned. But there could be two other reasons. On the one hand, it could be due to the fact that the decision rule \cref{CautiousTrigger} was not fulfilled in some iterations and therefore no new vectors were added to the memory. If this happens, it leads to a “worse” approximation, because the information about the curvature is not updated. With a memory size of $4$, this can have serious effects on the computation of the search direction $\eta_k$. If the search direction is not calculated “well”, it can happen that the determination of an appropriate stepsize turns out to be very complex. This, of course also, takes time. \\
On the other hand, it can also be due to the fact that the two methods could differ in one point, which concerns numerical accuracy. \cite{HuangGallivanAbsil:2015} state that it can happen, that some “old” pairs of $\{ \widetilde{s}_i, \widetilde{y}_i\}_{i=k-m}^{k-1}$ are not in the tangent space $\tangent{x_k}$ because of the repeated use of the chosen vector transport, $\vectorTransportSymbol^S$, which can be attributed to its numerical inaccuracy, if it is significant. \cite{HuangGallivanAbsil:2015} note that in order to avoid this disadvantage, one can project all vectors, which were transported more than once, i.e. $\{ \widetilde{s}_i, \widetilde{y}_i \}_{i=k-m+1}^{k-1}$, again onto the tangent space of the next iterate $\tangent{x_{k+1}}$ \cite[p.~1676]{HuangGallivanAbsil:2015}. Whether this approach is actually used in the implementation of the method from \cite{HuangGallivanAbsil:2015}, used for the experiments, is left open. However, the implementation of the CLRBFGS method from \lstinline!Manopt.jl! also differs from the actual LRBFGS method, \cref{LRBFGSMethod} (not only by using the cautious decision rule): Before \cref{LRBFGSTwo-LoopRecursion} returns the search direction $\eta_k$, it is projected again onto the tangent space of the current iterate, i.e. \cref{LRBFGSTwo-LoopRecursion} actually returns $\eta_k =  \proj{x_k}(- \mathcal{B}^{LRBFGS}_k[\operatorname{grad} f(x_k)]) \in \tangent{x_k}$. This is to make sure, that the search direction $\eta_k$, which may have been calculated with inaccuracies in \cref{LRBFGSTwo-LoopRecursion}, actually lies in the appropriate tangent space $\tangent{x_k}$. These different approaches of the two methods can of course be reflected both in the number of iterations and in the average time per run. If this is the case, the LRBFGS method from \cite{HuangGallivanAbsil:2015} is of course again to be preferred.

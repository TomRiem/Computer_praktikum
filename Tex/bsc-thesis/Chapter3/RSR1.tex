\section{Riemannian Symmetric Rank-One Update}

The direct SR1 update for operators on tangent spaces of the manifold which approximate the action of the Riemannian Hessian $\operatorname{Hess} f(x_{k+1}) [\cdot]$ is given by 
\begin{equation}\label{RiemannianDirectSR1formula}
    \mathcal{H}^\mathrm{SR1}_{k+1} [\cdot] = \widetilde{\mathcal{H}}^\mathrm{SR1}_k [\cdot] + \frac{(y_k - \widetilde{\mathcal{H}}^\mathrm{SR1}_k [s_k]) (y_k - \widetilde{\mathcal{H}}^\mathrm{SR1}_k [s_k])^{\flat} [\cdot] }{(y_k - \widetilde{\mathcal{H}}^\mathrm{SR1}_k [s_k])^{\flat} [s_k]}
\end{equation}
\cite[p.~18]{Huang:2013}, and it is not difficult to derive the update formula for the approximation of the Hessian inverse ${\operatorname{Hess} f(x_{k+1})}^{-1} [\cdot]$, which is given by 
\begin{equation}\label{RiemannianInverseSR1formula}
    \mathcal{B}^\mathrm{SR1}_{k+1} [\cdot] = \widetilde{\mathcal{B}}^\mathrm{SR1}_k [\cdot] + \frac{(s_k - \widetilde{\mathcal{B}}^\mathrm{SR1}_k [y_k]) (s_k - \widetilde{\mathcal{B}}^\mathrm{SR1}_k [y_k])^{\flat} [\cdot] }{(s_k - \widetilde{\mathcal{B}}^\mathrm{SR1}_k [y_k])^{\flat} [y_k]}.
\end{equation}
We see immediately that both \cref{RiemannianDirectSR1formula} and \cref{RiemannianInverseSR1formula} create self-adjoint operators if $\widetilde{\mathcal{H}}^\mathrm{SR1}_k$ or $\widetilde{\mathcal{B}}^\mathrm{SR1}_k$ are self-adjoint (which are self-adjoint if $\mathcal{H}^\mathrm{SR1}_k$ or $\mathcal{B}^\mathrm{SR1}_k$ are self-adjoint, since the parallel transport $\parallelTransportSymbol$ is an isometric vector transport). Furthermore, we see that the update formula \cref{RiemannianDirectSR1formula} produces a positive definite operator if $\widetilde{\mathcal{H}}^\mathrm{SR1}_k$ is positive definite (which is positive definite if $\mathcal{H}^\mathrm{SR1}_k$ is positive definite, since the parallel transport $\parallelTransportSymbol$ is an isometric vector transport) and $(y_k - \widetilde{\mathcal{H}}^\mathrm{SR1}_k [s_k])^{\flat} [s_k] = g_{x_{k+1}}(y_k - \widetilde{\mathcal{H}}^\mathrm{SR1}_k [s_k], s_k) >0$ holds (the same of course for \cref{RiemannianInverseSR1formula}). This shows that, just as in the Euclidean case, the search direction $\eta_k \in \tangent{x_k}$ does not always have to be a descent direction and that there is also the possibility that this method can break down if the numerator in \cref{RiemannianDirectSR1formula} or \cref{RiemannianInverseSR1formula} is equal to zero. In order to overcome this, \cref{CautiousSR1} can be transferred to the Riemannian setup. This means that \cref{RiemannianDirectSR1formula} is only executed if
\begin{equation}\label{RiemannianCautiousSR1}
    \lvert g_{x_{k+1}}(y_k - \widetilde{\mathcal{H}}^\mathrm{SR1}_k [s_k], s_k) \lvert \; \geq \; r \; \lVert s_k \rVert_{x_{k+1}} \lVert y_k - \widetilde{\mathcal{H}}^\mathrm{SR1}_k [s_k] \rVert_{x_{k+1}} 
\end{equation}
holds, where $g_{x_{k+1}}(\cdot, \cdot) \colon \; \tangent{x_{k+1}} \times \tangent{x_{k+1}} \to \mathbb{R}$ is the inner product on $\tangent{x_{k+1}}$ (see \cite[p.~6]{BergmannHerzogLouzeiroSilvaTenbrinckVidalNunez:2020:1}) and $ \lVert \cdot \rVert_{x_{k+1}} = \sqrt{g_{x_{k+1}}(\cdot, \cdot)}$ denotes the resulting norm on $\tangent{x_{k+1}}$. The factor $r$ is again in the interval $(0,1)$. In order to be able to apply this methodology to \cref{RiemannianInverseSR1formula} as well, it is checked in each iteration whether 
\begin{equation}\label{RiemannianCautiousSR1.2}
    \lvert g_{x_{k+1}}(s_k - \widetilde{\mathcal{B}}^\mathrm{SR1}_k [y_k], y_k) \lvert \; \geq \; r \; \lVert y_k \rVert_{x_{k+1}} \lVert s_k - \widetilde{\mathcal{B}}^\mathrm{SR1}_k [y_k] \rVert_{x_{k+1}} 
\end{equation}
is fulfilled. If this is the case, \cref{RiemannianInverseSR1formula} is executed. To get to \cref{RiemannianCautiousSR1.2}, the triple $(\widetilde{\mathcal{H}}^\mathrm{SR1}_k, s_k, y_k)$ in \cref{RiemannianCautiousSR1} was simply replaced by the triple $(\widetilde{\mathcal{B}}^\mathrm{SR1}_k, y_k, s_k)$. We see that by fulfilling \cref{RiemannianCautiousSR1.2} and if $\widetilde{\mathcal{B}}^\mathrm{SR1}_k$ is a positive definite operator, \cref{RiemannianInverseSR1formula} produces a positive definite operator. \\
In summary, a Riemannian inverse SR1 quasi-Newton method, short inverse RSR1 method, which uses the approximation of the Hessian inverse ${\operatorname{Hess} f(x_{k+1})}^{-1} [\cdot]$, is as follows:
\begin{algorithm}[H]
    \caption{Inverse SR1 Method}\label{RiemannianSR1Method}
    \begin{algorithmic}[1]
        \State Continuously differentiable real-valued function $f$ on $\mathcal{M}$, bounded below; initial iterate $x_0 \in \mathcal{M}$; initial positive definite and self-adjoint operator $\mathcal{B}^{\mathrm{SR1}}_0 \colon \; \tangent{x_0} \to \tangent{x_0}$; convergence tolerance $\varepsilon > 0$. Set $k = 0$.
        \While{$\lVert \nabla f(x_k) \rVert > \varepsilon$}
            \State Compute the search direction $\eta_k = - \mathcal{B}^{\mathrm{SR1}}_k [\operatorname{grad} f(x_k)]$.
            \State  Determine a suitable stepsize $\alpha_k > 0$. 
            \State Set $x_{k+1} = \exponential{x_k} (\alpha_k \eta_k)$.
            \State Set $\widetilde{\mathcal{B}}^{\mathrm{SR1}}_k = \parallelTransportDir{x_k}{\alpha_k \eta_k} \circ \mathcal{B}^{\mathrm{SR1}}_k \circ {\parallelTransportDir{x_k}{\alpha_k \eta_k}}^{-1}$, $s_k = \parallelTransportDir{x_k}{\alpha_k \eta_k} (\alpha_k \eta_k)$ and 
            \StatexIndent[2] $y_k = \operatorname{grad} f(x_{k+1}) - \parallelTransportDir{x_k}{\alpha_k \eta_k} (\operatorname{grad} f(x_k))$.
            \State Compute $\mathcal{B}^{\mathrm{SR1}}_{k+1} \colon \; \tangent{x_{k+1}} \to \tangent{x_{k+1}}$ by means of \cref{RiemannianInverseSR1formula}. 
            \State Set $k = k+1$.
        \EndWhile
        \State \textbf{Return} $x_k$.
    \end{algorithmic}
\end{algorithm}
The statements about the termination of quadratic functions and the generation of good approximations known from the Euclidean case have not yet been transferred to the Riemannian setup. Nevertheless, we can make the bold assumption that this is feasible under reasonable assumptions.  
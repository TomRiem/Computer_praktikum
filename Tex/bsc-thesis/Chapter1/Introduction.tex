\chapter{Introduction}

Optimization on Riemannian manifolds, also called Riemannian optimization, concerns finding an optimum of a real-valued function $f$ defined over a manifold. It can be thought of as unconstrained optimization on a constrained space. In this context, attention also turned to the generalization of quasi-Newton methods to the Riemannian setup. The first research paper to focus this topic was \cite{Gabay:1982}, which deals with the generalization of the well-known BFGS method, which has also received the most attention so far in the Riemannian setup. This work deals with the generalization of the not less unknown SR1 quasi-Newton update, which, in contrast to many other methods of this class, does not inherit the positive definiteness. It is shown how the SR1 update was generalized for operators on the tangent space of a manifold and the efficiency of a Riemannian SR1 quasi-Newton method implemented in Julia is compared with that of a Riemannian BFGS method.


\chapter{Introduction}

In the past decades, a notable interest has grown for the problem of minimizing a smooth objective function $f$ on a Riemannian manifold, which offers efficient alternative formulations to many problems. The applications are many and varied, they occur in engineering and science, which include the following fields: algorithmic questions pertaining to linear algebra, signal processing, data mining, statistical image analysis, financial mathematics, nanostructures, model reduction of dynamical systems and more. Optimization on Riemannian manifolds, also called Riemannian optimization, concerns finding an optimum of a real-valued function $f$ defined over a manifold. It can be thought of as unconstrained optimization on a constrained space. As such, optimization algorithms on manifolds are not fundamentally different from classical algorithms for unconstrained optimization in $\mathbb{R}^n$. On an Euclidean space, various methods of solving unconstrained optimization problems are known. The concepts of these algorithms can be used for the Riemannian optimization if many definitions are reconsidered. This reconsideration is crucial because the ideas are not extended simply from the Euclidean setup. The book \cite{AbsilMahonySepulchre:2008} provides a comprehensive introduction to this area, with an emphasis on providing the necessary background in differential geometry instrumental to algorithmic development. \\
Many manifold-based algorithms have been proposed or are under development. The reason for this is that they bring significant benefits, such as that all the iterates stay on the manifold, i.e., they satisfy the constraints (this property allows us to stop the iteration early), that they have the convergence properties of unconstrained optimization algorithms while operating on a constrained set, that there is no need to consider Lagrange multipliers or penalty functions and more \cite[p.~2-3]{Huang:2013}. The idea of quasi-Newton methods on Riemannian manifolds is also not new. The first research paper to focus this topic was \cite{Gabay:1982}, but it was barely noticed. Nevertheless, a generalization of quasi-Newton methods in general and the BFGS method in particular is becoming more and more popular, since the many positive properties can be transferred to the Riemannian setting. In the Euclidean setting, the BFGS method is a well-known quasi-Newton method that has been viewed for many years as the best quasi-Newton method for solving unconstrained optimization problems, therefore much attention has been paid to generalizing this method to Riemannian manifolds. \\
This thesis is intended to deal with the BFGS method on Riemannian manifolds. We are interested in whether, and above all, how the BFGS method can be generalized for the application on Riemannian manifolds. We want to summarize the currently known Riemannian BFGS methods. Their core aspects should be discussed and their convergence results should be presented. Furthermore, we are interested in how a BFGS method on Riemannian manifolds can be implemented efficiently and which requirements have to be taken into account. An implementation of such a method should happen and its performance should be compared with results of other BFGS methods on Riemannian manifolds. \\

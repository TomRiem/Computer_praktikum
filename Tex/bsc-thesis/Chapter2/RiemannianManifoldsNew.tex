\chapter{Riemannian Manifolds}
\label{Chapter3}

This chapter is dedicated to the introduction of the basic principles about Riemannian manifolds and resulting concepts, which we need for a reasonable derivation of numerical methods on these structures. The following material has been adopted from \cite{AbsilMahonySepulchre:2008}, \cite{Huang:2013}, \cite{Lee:1997}, \cite{Lee:2003} and \cite{Lee:2019}, unless otherwise noted. \\

\section{Smooth Manifolds}

Let $\mathcal{M}$ be a set. A bijection $\phi$ of a subset $U$ of $\mathcal{M}$ onto an open subset of $\mathbb{R}^n$ is called a $n$-dimensional coordinate chart, or just a chart, of the set $\mathcal{M}$, denoted by $(U, \phi)$. A ($C^{\infty}$) atlas of $\mathcal{M}$ into $\mathbb{R}^n$ is a collection of charts $\{ (U_{\alpha}, \phi_{\alpha}) \}_{\alpha \in A}$ of the set $\mathcal{M}$ such that
\begin{enumerate}
    \item $\mathcal{M} \subseteq \bigcup_{\alpha \in A} U_{\alpha}$
    \item for any pair $\alpha, \beta \in A$ with $U_{\alpha} \cap U_{\beta} \neq \emptyset$, the sets $\phi_{\alpha}(U_{\alpha} \cap U_{\beta})$ and $\phi_{\beta}(U_{\alpha} \cap U_{\beta})$ are open sets in $\mathbb{R}^n$ and the change of coordinates \begin{equation*}
     \phi_{\beta} \circ \phi_{\alpha}^{-1} \colon \; \mathbb{R}^n \to \mathbb{R}^n \end{equation*} is smooth (i.e. is element of $C^{\infty}$, differentiable for all degrees of differentiation) on its domain $\phi_{\alpha}(U_{\alpha} \cap U_{\beta})$. We say that the elements
    of an atlas overlap smoothly.
\end{enumerate}
We identify an atlas $\{ (U_{\alpha}, \phi_{\alpha}) \}_{\alpha \in A}$ with its index set $A$. Two atlases $A_1$ and $A_2$ are equivalent if $A_1 \cup A_2$ is an atlas. Given an atlas $A$, let $A^{+}$ be the set of all charts $(U, \phi)$ such that $A \cup (U, \phi)$ is also an atlas. $A^{+}$ is an atlas, called the maximal atlas (or complete atlas) generated by the atlas $A$. A maximal atlas of a set $\mathcal{M}$ is also called a differentiable structure on $\mathcal{M}$. A smooth ($n$-dimensional) manifold is a couple $(\mathcal{M}, A^{+})$, where $\mathcal{M}$ is a set and $A^{+}$ is a maximal smooth ($C^{\infty}$) atlas of $\mathcal{M}$ into $\mathbb{R}^n$, such that the topology induced by $A^{+}$ is Hausdorff and second-countable \cite[18-20]{AbsilMahonySepulchre:2008}. Hausdorff means every single point is a closed set and second-countable means there is a countable collection of open sets that generates all open sets by union \cite[p.~4]{Huang:2013}. \\
For a smooth manifold $(\mathcal{M}, A^{+})$, every coordinate chart in the given maximal atlas is called a smooth coordinate chart for $\mathcal{M}$ or just a smooth chart. Given a smooth chart $(U, \phi)$, the component functions of $\phi$ are called local coordinates, or just coordinates, for $\mathcal{M}$, and are written as $(x^1, \cdots, x^n)$, that means, $x^i$ returns the $i$-th element of the vector $\phi(x) \in \mathbb{R}^n$ for $x \in U$. \cite[p.~374-375]{Lee:2019}. \\
Let $\mathcal{M}$ be a smooth ($n$-dimensional) manifold and $f \colon \; \mathcal{M} \to \mathbb{R}$ a real-valued function. We say that $f$ is a smooth function if for every $x \in \mathcal{M}$ there exists a smooth chart $(U, \phi)$ for $\mathcal{M}$ whose domain contains $x$ and such that the composite function $f \circ \phi^{-1}$ is smooth on the open subset $\phi(U) \subseteq \mathbb{R}^n$. The set of all smooth functions on $\mathcal{M}$ is denoted by $C^{\infty}(\mathcal{M})$. Because sums and constant multiples of smooth functions are smooth, $C^{\infty}(\mathcal{M})$ is a vector space over $\mathbb{R}$ \cite[p.~32-33]{Lee:2003}. \\

\section{Tangent Vectors}

In order to apply optimization algorithms based on line search, we must consider directions on a manifold. For that we introduce the tangent space to a manifold at a point, which can be interpreted as a “linear model” for the manifold near this point. There are various equivalent ways of defining tangent vectors on a smooth manifold $\mathcal{M}$. Tangent vectors are often considered either as derivations of $C^{\infty}$-functions at $x$ on $\mathcal{M}$, or as equivalence classes of curves through $x$ under a suitable equivalence relation \cite[p.~15]{Lee:1997}. \\
First the latter approach:

\begin{definition}[{\cite[Definition~3.5.1]{AbsilMahonySepulchre:2008}}]
    A tangent vector $\xi_x$ to a manifold $\mathcal{M}$ at a point $x$ is a map from $C^{\infty}_x(\mathcal{M})$ (the set of smooth real-valued functions defined on a neighborhood of $x$) to $\mathbb{R}$ such that there exists a curve $\geodesicSymbol$ on $\mathcal{M}$ with $\geodesic<s>(0) = x$, satisfying

    \begin{equation*}
        \xi_x (f) = \dot{\geodesicSymbol}(0) \; f = \frac{\mathrm{d}}{\mathrm{d}t} f(\geodesic<s>(t)) \; \big{\vert}_{t=0}
    \end{equation*}
    for all $f \in C^{\infty}_x(\mathcal{M})$. Such a curve $\geodesicSymbol$ is said to realize the tangent vector $\xi_x$.
\end{definition}

The point $x$ is called the foot of the tangent vector $\xi_x$. Given a tangent vector $\xi_x$ to $\mathcal{M}$, there are infinitely many curves $\geodesicSymbol$ that realize $\xi_x$ (i.e. $\geodesic<s>(0) = 0$ and $\dot{\geodesicSymbol}(0) = \xi_x$). They can be characterized as follows 

\begin{proposition}[{\cite[Proposition~3.5.2]{AbsilMahonySepulchre:2008}}]
    Two curves $\geodesicSymbol_1$ and $\geodesicSymbol_2$ through a point $x$ at $t = 0$ satisfy $\dot{\geodesicSymbol}_1(0) = \dot{\geodesicSymbol}_2(0)$ if and only if, given a chart $(U, \phi)$ with $x \in U$, it holds that

    \begin{equation*}
        \frac{\mathrm{d}}{\mathrm{d}t} (\phi \circ \geodesicSymbol_1)\; \vert_{t = 0} = \frac{\mathrm{d}}{\mathrm{d}t} (\phi \circ \geodesicSymbol_2)\; \vert_{t = 0} 
    \end{equation*}\end{proposition}

That means a tangent vector at the point $x$ can be seen as an equivalence class $[\geodesicSymbol]$ of these curves. Thus tangent vectors can be understood as “velocities”. \\
The other definition is the most convenient to work with in practice. For every point $x \in \mathcal{M}$, a tangent vector at $x$ is a $\mathbb{R}$-linear map $\xi_x \colon \; C^{\infty}(\mathcal{M}) \to \mathbb{R}$ that is a derivation at $x$, meaning that for all $f, g\in C^{\infty}(\mathcal{M})$ it satisfies the Leibniz rule (product rule):
\begin{equation}\label{LeibnizRule}
    \xi_x(fg) = f(x) \; \xi_x(g) + g(x) \; \xi_x(f).
\end{equation}
The set of all tangent vectors at $x$ is denoted by $\tangent{x}$ and called the tangent space at $x$. From now on Elements of $\tangent{x}$ will be denoted by $\xi_x$ and $\eta_x$ etc. or simply $\xi$ and $\eta$ when the base point is clear from the context. \\
The tangent space admits a vector space structure, i.e. for all $\xi_x, \eta_x \in \tangent{x}$ and $a, b \in \mathbb{R}$ it holds that
\begin{equation*}
    a \; \xi_x + b \; \eta_x \in \tangent{x}.
\end{equation*}
The property that the tangent space $\tangent{x}$ is a vector space is very important. In the same way that the derivative of a real-valued function provides a local linear approximation of the function, the tangent space $\tangent{x}$ provides a local vector space approximation of the manifold \cite[p.~34]{AbsilMahonySepulchre:2008}. \\
Let $\mathcal{M}$ be a smooth $n$-dimensional manifold and $(U, \phi)$ a smooth chart containing $x \in \mathcal{M}$. Let $(x^1, \cdots, x^n)$ be the smooth local coordinate functions of $\phi$. We define the coordinate vectors $\partial / \partial x^1 \; \vert_x, \cdots, \partial / \partial x^n \; \vert_x$ at $x$ by 
\begin{equation*}
    \frac{\partial}{\partial x^i} \; \bigg{\vert}_x \; f = \frac{\partial}{\partial x^i} \; \bigg{\vert}_{\phi(x)} (f \circ \phi^{-1}).
\end{equation*}
These vectors form a basis for $\tangent{x}$, which therefore has $\tangent{x}$ also dimension $n$. Thus once a smooth chart has been chosen, every tangent vector $\xi_x \in \tangent{x}$ can be written uniquely in the form
\begin{equation*}
    \xi_x = \sum^{n}_{i = 1} \xi^{i}_x \; \frac{\partial}{\partial x^i} \; \bigg{\vert}_x
\end{equation*}
where the components $\xi^{1}_x, \cdots, \xi^{n}_x$ are obtained by applying $\xi_x$ to the coordinate functions, i.e. $\xi^{i}_x = \xi_x(x^i)$. If it does not allow confusion, the abbreviation $\partial / \partial x^i \; \vert_x = \partial_i \; \vert_x$ is used \cite[p.~376-377]{Lee:2019}. It follows that we can represent the tangent vector $\xi_x \in \tangent{x}$ with respect to the basis $(\partial_1 \; \vert_x, \cdots, \partial_n \; \vert_x)$ in real-valued coordinates, i.e. $\hat{\xi}_x = (\xi^{1}_x, \cdots, \xi^{n}_x) \in \mathbb{R}^n$. \\
Given a smooth $n$-dimensional manifold $\mathcal{M}$, we define the tangent bundle of $\mathcal{M}$, denoted by $\tangent{}$ as the disjoint union of the tangent spaces at all points of $\mathcal{M}$:
\begin{equation*}
    \tangent{} = \coprod_{p \in \mathcal{M}} \tangent{p}.
\end{equation*}
The tangent bundle $\tangent{}$ can be thought of both as a union of vector spaces and as a smooth manifold of dimension $2n$. \\
A vector field on a manifold $\mathcal{M}$ is a smooth map $\xi \colon \; \mathcal{M} \to \tangent{}$ which assigns to each point $x \in \mathcal{M}$ a tangent vector $\xi_x \in \tangent{x}$. Given a vector field $\xi$ on $\mathcal{M}$ and a smooth function $f \in C^{\infty}(\mathcal{M})$, we let $\xi \; f$ denote the real-valued function on $\mathcal{M}$ defined by
\begin{equation*}
    (\xi \; f)(x) = \xi_x (f)
\end{equation*}
for all $x$ in $\mathcal{M}$. The multiplication of a vector field by a function $f \in C^{\infty}(\mathcal{M})$ and the addition of two vector fields are defined as follows:
\begin{align*}
    (\xi \; f)_x & = f(x) \; \xi_x \\
    (\xi + \zeta)_x & = \xi_x + \zeta_x
\end{align*}
for all $x \in \mathcal{M}$. Smoothness is preserved by these operations. We let $\mathfrak{X}(\mathcal{M})$ denote the set of smooth vector fields endowed with these two operations \cite[p.~34]{AbsilMahonySepulchre:2008}. \\
If $(U, \phi)$ is a smooth chart for $\mathcal{M}$ with smooth coordinate functions $(x^1, \cdots, x^n)$, for each $i$ we get a smooth coordinate vector field $E_i$ on $U$, denoted by $E_i = \frac{\partial}{\partial x^i}= \partial_i$ and it is called the $i$-th coordinate vector field of $(U, \phi)$. Its value at $x \in U$ is the coordinate vector $\frac{\partial}{\partial x^i} \;\vert_x \in \tangent{x}$. These coordinate vector fields are smooth, and every vector field $\xi \colon \; \mathcal{M} \to \tangent{}$ admits the decomposition
\begin{equation*}
    \xi = \sum^{n}_{i=1} (\xi x^i) E_i = \sum^{n}_{i=1} \xi^i E_i
\end{equation*}
on $U$ \cite[p.~36-37]{AbsilMahonySepulchre:2008}. The vector fields $(E_1, \cdots, E_n)$ form a smooth local frame for $\tangent{}$, called a coordinate frame \cite[p.~384-385]{Lee:2019}. \\
For each $x \in \mathcal{M}$ we define the cotangent space at $x$, denoted by $\cotangent{x}$, to be the dual space to $\tangent{x}$, i.e. $\cotangent{x} = (\tangent{x})^{*}$. It is the space of all real-valued linear functionals on $\tangent{x}$, which are called tangent covectors or cotangent vectors at $x$. For every $f \in C^{\infty}(\mathcal{M})$ and $x \in \mathcal{M}$, there is a covector $\mathrm{d} f_x \in \cotangent{x}$ called the differential of $f$ at $x$, defined by 
\begin{equation*}
    \mathrm{d} f_x (\xi_x) = \xi_x(f)
\end{equation*}
for all $\xi_x \in \tangent{x}$ \cite[p.~377]{Lee:2019}. \\

\section{Riemannian Metric}
\label{Section3.3}

A Riemannian metric on a smooth manifold $\mathcal{M}$ is a smooth covariant 2-tensor field $g \in \mathfrak{T}(\mathcal{M})$ (cf. \cite[p.~397]{Lee:2019}) whose value $g_x$ at each $x \in \mathcal{M}$ is an inner product on $\tangent{x}$. This means $g$ is a symmetric 2-tensor field that is positive definite in the sense that $g_x(\xi_x, \xi_x) \geq 0$ for each $x \in \mathcal{M}$ and each $\xi_x \in \tangent{x}$, with equality if and only if $\xi_x = 0_x$, i.e. $\xi_x$ is the zero tangent vector in $\tangent{x}$ \cite[p.~11]{Lee:2019}. It is therefore a structure on $\mathcal{M}$ that is understood as a smoothly varying family of inner products on the tangent spaces $\tangent{x}$ \cite[p.~6]{BergmannHerzogLouzeiroSilvaTenbrinckVidalNunez:2020:1}. \\
A Riemannian manifold is a pair $(\mathcal{M},g)$ where $\mathcal{M}$ is a smooth manifold and $g$ is a specific choice of Riemannian metric on $\mathcal{M}$ \cite[p.~11]{Lee:2019}. From now on we assume that $\mathcal{M}$ is a Riemannian manifold and we denote the inner product on the tangent space $\tangent{x}$ for each $x \in \mathcal{M}$ with
\begin{align*}
    g_x \colon \; \tangent{x} \times \tangent{x} & \to \mathbb{R} \\
    (\xi_x, \eta_x) & \mapsto g_x(\xi_x, \eta_x)
\end{align*}
Using this inner product, we can define the norm of tangent vectors, angles between nonzero tangent vectors, and orthogonality of tangent vectors:
\begin{enumerate}
    \item The length or norm of a tangent vector $\xi_x \in \tangent{x}$ is defined as \begin{equation*}
     \lVert \xi_x \rVert_x = \sqrt{g_x(\xi_x, \xi_x)}. \end{equation*}    \item The angle between two nonzero tangent vectors $\xi_x, \eta_x \in \tangent{x}$ is the unique $\theta \in [0, \pi]$ satisfying \begin{equation*}
     \cos(\theta) = \frac{g_x(\xi_x, \eta_x)}{\lVert \xi_x \rVert_x \lVert \eta_x \rVert_x}. \end{equation*}    \item Tangent vectors $\xi_x, \eta_x \in \tangent{x}$ are said to be orthogonal if $g_x(\xi_x, \eta_x) = 0$. This means either one or both vectors are zero, or the angle between them is $\pi / 2$.
\end{enumerate}
We denote by 
\begin{equation*}
    \mathbb{B}(0_x, r) = \{ \xi_x \in \tangent{x} \colon \; \lVert \xi_x \rVert_x < r \}
\end{equation*}
the ball of radius $r > 0$ in $\tangent{x}$ centered at the origin $0_x$. \\
Let $(U, \phi)$ be a chart of a Riemannian manifold $(\mathcal{M},g)$. The components of $g$ in the chart $\phi$ are given by
\begin{equation*}
    g_{ij} = g(E_i, E_j)
\end{equation*}
where $E_i = \frac{\partial}{\partial x^i}= \partial_i$ the $i$-th coordinate vector field. Thus, for vector fields $\xi = \sum^{n}_{i=1} \xi^i E_i$, $\eta = \sum^{n}_{i=1} \eta^i E_i$, we have 
\begin{equation*}
    g(\xi, \eta) = \sum^{n}_{i, j=1} g_{ij} \xi^i \eta^j.
\end{equation*}
Note that the $g_{ij}$'s are $n^2$ smooth functions $g_{ij} \colon \; \mathcal{M} \to \mathbb{R}$ for $i,j = 1, \cdots, n$. \\
These real-valued functions can be summarized as a matrix-valued function $G_{\cdot} = (g_{ij}(\cdot))$, characterized by $(G_x)_{ij} = g_{ij} (x) = g_x(E_i \vert_x, E_j \vert_x)$. The matrix $G_x$ depends smoothly on $x \in U$, is symmetric in $i$ and $j$ and positive definite at every point $x \in \mathcal{M}$. If $\xi_x = \sum^{n}_{i = 1} \xi^{i}_x E_i \vert_x$ is a vector in $\tangent{x}$ such that $\sum^{n}_{i = 1} g_{ij} (x) \xi^{i}_x = 0$, it follows that $g_x(\xi_x, \xi_x) = 0$, which implies $\xi_x = 0_x$. Thus the matrix $(g_{ij} (p))$ is always nonsingular \cite[p.~13]{Lee:2019}. This means that one is able to express the inner product of two tangent vectors $\xi_x, \eta_x \in \tangent{x}$ by $g_x (\xi_x, \eta_x) = \hat{\xi}^{\mathrm{T}}_x G_x \hat{\eta}_x$, where $\hat{\xi}_x, \hat{\eta}_x \in \mathbb{R}^n$ are the coordinate expression of the tangent vectors with respect to the basis $(E_1 \vert_x, \cdots, E_n \vert_x)$. Thus a linear operator $\mathcal{B}$ on a tangent space and a vector transport $\vectorTransportSymbol$ admit matrix expressions $\hat{\mathcal{B}}$ and $\hat{\vectorTransportSymbol}$ that are called coordinate expressions. Without loss of generality, one can always choose the orthonormal vector fields $(\overline{E}_1, \cdots, \overline{E}_n)$. In this case, the matrix expression of $G_x$ is the identity and $\lVert \eta_x \rVert_x = \sqrt{g_x(\eta_x, \eta_x)} = \sqrt{\hat{\eta}^{\mathrm{T}}_x \hat{\eta}_x} = \lVert \hat{\eta}_x \rVert_2$, where $\lVert \cdot \rVert_2$ denotes the Euclidean norm \cite[p.~11]{Huang:2013}. \\ 
The most important tool that a Riemannian metric is the ability to measure lengths of curves and distances between points. If $\mathcal{M}$ is a smooth manifold, a (continuous) curve segment $\geodesicSymbol \colon \; [a,b] \to \mathcal{M}$ is said to be piecewise regular if there exists a partition $(a_0, \cdots, a_k)$ of $[a,b]$ (a finite sequence of real numbers such that $a = a_0 < a_1 < \cdots < a_k = b$) such that $\geodesicSymbol \vert_{[a_{i-1}, a_i]}$ is a regular curve segment (i.e. a smooth curve $\geodesicSymbol \colon \; [a_{i-1}, a_i] \to \mathcal{M}$ with $\dot{\geodesic<s>}(t) \neq 0$ for $t \in [a_{i-1}, a_i]$) for $i = 1, \cdots, k$. We refer to a piecewise regular curve segment as an admissible curve. The length of an admissible curve $\geodesicSymbol \colon \; [a,b] \to \mathcal{M}$ is defined as
\begin{equation*}
    \mathrm{L}_g(\geodesicSymbol) = \int^{b}_a \lVert \dot{\geodesic<s>}(t) \rVert_{\geodesic<s>(t)} \mathrm{d}t = \int^{b}_a \sqrt{ g_{\geodesic<s>(t)} (\dot{\geodesic<s>}(t), \dot{\geodesic<s>}(t)) } \mathrm{d}t.
\end{equation*}
The integrand is bounded and continuous everywhere on $[a,b]$ except possibly at the finitely many points where $\geodesicSymbol$ is not smooth, so this integral is well defined \cite[p.~33-34]{Lee:2019}. \\
For each pair of points $x, y \in \mathcal{M}$, we define the Riemannian distance from $x$ to $y$, denoted by $\operatorname{dist}(x,y)$, to be the infimum of the lengths of all admissible curves from $x$ to $y$, i.e.
\begin{equation*}
    \operatorname{dist}(x,y) = \inf_{\geodesicSymbol} \bigl\{ \int^{b}_a \lVert \dot{\geodesic<s>}(t) \rVert_{\geodesic<s>(t)} \mathrm{d}t \bigr\} = \inf_{\tilde{\geodesicSymbol}} \bigl\{ \int^{1}_0 \lVert \dot{\tilde{\geodesic<s>}}(t) \rVert_{\tilde{\geodesic<s>}(t)} \mathrm{d}t \bigr\}
\end{equation*}
where $\tilde{\geodesicSymbol}$ is a reparametrization of $\geodesicSymbol$ (cf. \cite[p.~34]{Lee:2019}) with $\geodesic<s>(0) = x$ and $\geodesic<s>(1) = y$. Assuming (as usual) that $\mathcal{M}$ is Hausdorff, it can be shown that the Riemannian distance defines a metric. We denote by
\begin{equation*}
    B_r(x) = \{ y \in \mathcal{M} \colon \; \operatorname{dist}(x,y) < r \}
\end{equation*}
the open metric ball of radius $r > 0$ with center $x \in \mathcal{M}$. \\
The Riemannian metric furnishes a linear bijective correspondence between the tangent and cotangent spaces via the index raising and index lowering functions or the so-called musical isomorphisms. This property of every Riemannian metric provides a natural isomorphism between the tangent and cotangent spaces which allows us to convert vectors to covectors and vice versa. Given a metric $g$ on $\mathcal{M}$, define a map called flat by
\begin{equation}\label{MusicallyIsomorphism}
    \flat \colon \; \tangent{x} \ni \xi_x \mapsto \xi^{\flat}_x \in \cotangent{x}
\end{equation}
satisfying
\begin{equation*}
    \xi^{\flat}_x(\eta_x) = g_x(\xi_x,\eta_x)
\end{equation*}
for all $\eta_x \in \tangent{x}$ and its inverse
\begin{equation*}
    \sharp \colon \; \cotangent{x} \ni \xi^{*}_x \mapsto \xi^{\sharp}_x \in \tangent{x}
\end{equation*}
satisfying
\begin{equation*}
    g_x(\xi^{\sharp}_x,\eta_x) = \xi^{*}_x(\eta_x) 
\end{equation*}
for all $\eta_x \in \tangent{p}$. The $\sharp$-isomorphism further introduces an inner product and an associated norm on the cotangent space $\cotangent{x}$, which we will also denote by $g_x(\cdot, \cdot)$ and $\lVert \cdot \rVert_x$, since it is clear which inner product or norm we refer to based on the respective arguments \cite[p.~6]{BergmannHerzogLouzeiroSilvaTenbrinckVidalNunez:2020:1}. \\
The most important application of the $\sharp$-isomorphism is to extend the classical gradient operator to Riemannian manifolds. If $(\mathcal{M}, g)$ is a Riemannian manifold and $f \colon \; \mathcal{M} \to \mathbb{R}$ is a smooth function, the gradient of $f$ is the vector field $\operatorname{grad} f = (\mathrm{d} f)^{\sharp}$ obtained from $\mathrm{d} f$ by raising an index. Unwinding the definitions, we see that $\operatorname{grad} f$ is characterized by the fact that
\begin{equation*}
    \mathrm{d} f_x (\xi) = g_x(\operatorname{grad} f \; \vert_x, \xi_x)
\end{equation*}
for all $x \in \mathcal{M}$, $\xi_x \in \tangent{x}$ \cite[p.~26-27]{Lee:2019}. For convenient reasons, we write $\operatorname{grad} f \; \vert_x = \operatorname{grad} f(x)$ for all $x \in \mathrm{M}$. \\


\section{Riemannian Connection, Riemannian Hessian, Geodesics, Retractions, Exponential Map, Parallel Transport and Vector Transports}
\label{Section3.4}

An analogy to a second derivative is required to define acceleration and, thereby, to generalize the Euclidean notion of a straight line between two points as being the one with zero acceleration. Likewise, the “straight” line on a Riemannian manifold, called a geodesic, is a curve $\geodesic<s>(t)$ that has zero acceleration. To define acceleration, we need a differentiation operator applicable to tangent vectors in different tangent spaces since $\dot{\geodesic<s>}(t)$ is a vector field along the curve. On Riemannian manifolds, differential operators are called affine connections \cite[p.~6]{Huang:2013}.\\
An affine connection $\nabla$ on a manifold $\mathcal{M}$ is a map 
\begin{align*}
    \nabla \colon \; \mathfrak{X}(\mathcal{M}) \times \mathfrak{X}(\mathcal{M}) & \to \mathfrak{X}(\mathcal{M}) \\
    (\eta, \xi) & \mapsto \nabla_{\eta} \xi
\end{align*}
which satisfies the following:
\begin{enumerate}
    \item $C^{\infty}(\mathcal{M})$-linearity in $\eta$: $\nabla_{f \eta + g \chi} \xi = f \; \nabla_{\eta} \xi + g \; \nabla_{\chi} \xi$,
    \item $\mathbb{R}$-linearity in $\xi$: $\nabla_{\eta} (a \; \xi + b \; \zeta) = a \; \nabla_{\eta} \xi + b \; \nabla_{\eta} \zeta$,
    \item Product rule (Leibniz rule): $\nabla_{\eta} (f \xi) = (\eta f) \xi + f \nabla_{\eta} \xi$,
\end{enumerate}
where $\eta, \xi, \chi, \zeta \in \mathfrak{X}(\mathcal{M})$, $f,g \in C^{\infty}(\mathcal{M})$ and $a,b \in \mathbb{R}$. The vector field $\nabla_{\eta} \xi$ is called the covariant derivative of $\xi$ with respect to $\eta$ for the affine connection $\nabla$ \cite[p.~94]{AbsilMahonySepulchre:2008}. At a point $x$ on $\mathcal{M}$, the connection maps tangent vectors $(\eta_x, \xi_x) \in \tangent{x} \times \tangent{x}$ to a tangent vector $\nabla_{\eta_x} \xi_x \in \tangent{x}$. The result $\nabla_{\eta_x} \xi_x$ is a covariant derivative of $\xi_x$ with respect to $\eta_x$. For a general manifold $\mathcal{M}$, there is an infinite number of affine connections. For a Riemannian manifold $(\mathcal{M}, g)$, one of the affine connections, called the Riemannian connection, uniquely satisfies two additional conditions \cite[p.~6]{Huang:2013}. Let $\langle \cdot, \cdot \rangle = g_{\cdot}(\cdot, \cdot)$ denote the Riemannian metric:

\begin{theorem}[{\cite[Theorem~5.3.1~(Levi-Civita)]{AbsilMahonySepulchre:2008}}]\label{RiemannianConnection}
    On a Riemannian manifold $\mathcal{M}$ there exists a unique affine connection $\nabla$ that satisfies
    \begin{enumerate}
        \item $\nabla_{\eta} \xi - \nabla_{\xi} \eta = [\eta, \xi]$ (symmetry),
        \item $\chi \langle \eta, \xi \rangle = \langle \nabla_{\chi} \eta, \xi \rangle + \langle \eta, \nabla_{\chi} \xi \rangle$ (compatibility with the Riemannian metric),
    \end{enumerate}
    for all $\eta, \xi, \chi \in \mathfrak{X}(\mathcal{M})$. This affine connection $\nabla$, called the Levi-Civita connection or the Riemannian connection of $\mathcal{M}$, is characterized by the Koszul formula
    \begin{equation*}
        2 \langle \nabla_{\chi} \eta, \xi \rangle = \chi \langle \eta, \xi \rangle + \eta \langle \xi, \chi \rangle - \xi \langle \chi, \eta \rangle - \langle \chi, [\eta, \xi] \rangle + \langle \eta, [\xi, \chi] \rangle + \langle \xi, [\chi, \eta] \rangle,
    \end{equation*}
    where $[\cdot, \cdot]$ denotes the Lie bracket (cf. \cite[p.~96-97]{AbsilMahonySepulchre:2008}).
\end{theorem}

Recall that for smooth vector fields $\eta, \xi, \chi \in \mathfrak{X}(\mathcal{M})$, $\langle \eta, \xi \rangle$ is a smooth real-valued function on $\mathcal{M}$ and $\chi \langle \eta, \xi \rangle$ is the smooth real-valued function given by the application of the vector field (i.e. derivation) $\chi$ to $\langle \eta, \xi \rangle$. \\
With the Riemannian connection the concept of the second derivative of a real-valued function $f \colon \; \mathcal{M} \to \mathbb{R}$ can be formalized. The Hessian can be understood as an operator acting on geometric objects and returning geometric objects. On an arbitrary Riemannian manifold, the Hessian operator is generalized as follows: \newpage

\begin{definition}[{\cite[Definition~5.5.1]{AbsilMahonySepulchre:2008}}]
    Given a real-valued function $f$ on a Riemannian manifold $\mathcal{M}$, the Riemannian Hessian of $f$ at a point $x$ in $\mathcal{M}$ is the linear map $\operatorname{Hess} f(x)$ of $\tangent{x}$ into itself defined by
    \begin{equation*}
        \operatorname{Hess} f(x) [\xi_x] = \nabla_{\xi_x} \operatorname{grad} f(x)
    \end{equation*}
    for all $\xi_x$ in $\tangent{x}$, where $\nabla$ is the Riemannian connection on $\mathcal{M}$.
\end{definition}

The Hessian has the following characteristic

\begin{proposition}[{\cite[Proposition~5.5.3]{AbsilMahonySepulchre:2008}}]
    The Riemannian Hessian is symmetric (in the sense of the Riemannian metric). That is, 
    \begin{equation*}
        g(\operatorname{Hess} f [\xi], \eta ) = g(\xi, \operatorname{Hess} f [\eta] )
    \end{equation*}
    for all $\xi, \eta \in \mathfrak{X}(\mathcal{M})$.
\end{proposition}

This means for $x \in \mathcal{M}$ and $\xi_x, \eta_x \in \tangent{x}$ we have $g_x(\operatorname{Hess} f(x) [\xi_x], \eta_x ) = g_x(\xi_x, \operatorname{Hess} f(x) [\eta_x] )$. The term “symmetric” could be misleading in some places. Since $\operatorname{Hess} f(x)$ can be seen as linear operator on $\tangent{x}$, we call it self-adjoint instead. \\
A geodesic $\geodesicSymbol$ on a manifold $\mathcal{M}$ endowed with an affine connection $\nabla$ is a curve with zero acceleration:
\begin{equation*}
    \nabla_{\dot{\geodesic<s>}(t)} \dot{\geodesic<s>}(t) = \frac{\mathrm{D}^2}{\mathrm{d}t^2} \geodesic<s>(t) = \frac{\mathrm{D}}{\mathrm{d}t} \dot{\geodesic<s>}(t) = 0
\end{equation*}
for all $t$ in the domain of $\geodesicSymbol$. Note that different affine connections produce different geodesics \cite[p.~102]{AbsilMahonySepulchre:2008}. A consequence of the compatibility with the Riemannian metric is that when the affine connection is the Riemannian connection, one of the geodesics between two points on the manifold (there may be many) is also a minimal length curve, which means a curve with minimal length between two points of $\mathcal{M}$ is always a monotone reparametrization of a geodesic relative to the Riemannian connection. These curves are called minimizing geodesics. This is consistent with the straight line in Euclidean space \cite[p.~7]{Huang:2013}. We denote by $\geodesic<l>{x}{\xi_x} \colon I \to \mathcal{M}$, with $I \subset \mathbb{R}$ being an open interval containing $0$, a geodesic starting at $x$ with $\dot{\geodesicSymbol}(0; x \, , \, \xi_x) = \xi_x$ for some $\xi_x \in \tangent{x}$. \\
In the Euclidean case, the concept of moving in the direction of a vector is straightforward. On a manifold, the notion of moving in the direction of a tangent vector, while staying on the manifold, is generalized by the notion of a a so-called retraction:

\begin{definition}[{\cite[Definition~4.1.1]{AbsilMahonySepulchre:2008}}]\label{Retraction}
    A retraction on a manifold $\mathcal{M}$ is a smooth map $\retractionSymbol \colon \; \tangent{} \to \mathcal{M}$ with the following properties. Let $\retract{x}(\cdot)$ denote the restriction of $\retractionSymbol$ to $\tangent{x}$. 
    \begin{enumerate}
        \item $\retract{x}(0_x) = x$, where $0_x$ denotes the zero element of $\tangent{x}$. 
        \item With the canonical identification $\tangent{0_x}[\tangent{x}] \simeq \tangent{x}$, $\retract{x}$ satisfies \begin{equation} \mathrm{D} \; \retract{x}(0_x) = \id_{\tangent{x}}, \label{LocalRigidity} \end{equation} where $\id_{\tangent{x}}$ denotes the identity map on $\tangent{x}$.  
    \end{enumerate}
\end{definition}

It is in general assumed that the domain of $\retractionSymbol$ is the whole tangent bundle $\tangent{}$. The condition \cref{LocalRigidity} is called local rigidity condition. Equivalently, for every tangent vector $\xi_x$ in $\tangent{x}$, the curve $\geodesicSymbol \colon \; t \to \retract{x}(t \; \xi_x)$ satisfies $\dot{\geodesicSymbol}(0) = \xi_x$. Moving along this curve $\geodesicSymbol$ is thought of as moving in the direction $\xi_x$ while constrained to the manifold $\mathcal{M}$. \\
The second important purpose of a retraction $\retract{x}$ is to transform functions defined in a neighborhood of $x \in \mathcal{M}$ into functions defined on the vector space $\tangent{x}$. Given a function $f \colon \; \mathcal{M} \to \mathbb{R}$ and $x \in \mathcal{M}$, we let $\hat{f}_x (\cdot) = f \circ \retract{x}(\cdot)$ denote the pullback of $f$ through $\retractionSymbol$ in $\tangent{x}$, which is a real-valued function on a vector space. Because \cref{LocalRigidity} holds, we have $\mathrm{D} \hat{f}_x (0_x) = \mathrm{D} f(x)$. If $\mathcal{M}$ is endowed with a Riemannian metric (and thus $\tangent{x}$ with an inner product), then
\begin{equation*}
    \operatorname{grad} \hat{f}_x (0_x) = \operatorname{grad} f(x)
\end{equation*}
holds \cite[p.~55-56]{AbsilMahonySepulchre:2008}. \\ 
Every Riemannian manifold has a special retraction called the Riemannian exponential map, which is, in the geometric sense, the most natural retraction to use on a Riemannian manifold. Given a point $x \in \mathcal{M}$ and a tangent vector $\xi_x \in \tangent{x}$, there is an interval $I$ containing $0$ and a unique geodesic $\geodesic<l>{x}{\xi_x} \colon I \to \mathcal{M}$ satisfying $\geodesic<l>{x}{\xi_x}(0) = x$ and $\dot{\geodesicSymbol}(0; x \, , \, \xi_x) = \xi_x$. In addition, this geodesic satisfies the homogeneity property, $\geodesic<l>{x}{a \; \xi_x}(t) = \geodesic<l>{x}{\xi_x}(at)$. The map 
\begin{align}\label{ExponentialMap}
    \exponential{x} \colon \; \tangent{x} & \to \mathcal{M} \\
    \xi_x & \mapsto \exponential{x}(\xi_x) = \geodesic<l>{x}{\xi_x}(1)
\end{align}
is called the exponential map at $x$. When the domain of definition of $\exponential{x}$ is the whole $\tangent{x}$ for all $x \in \mathcal{M}$, the manifold $\mathcal{M}$ is called (geodesically) complete. It can be shown that $\exponential{x}$ defines a diffeomorphism of a neighborhood $\hat{U}$ of the origin $0_x \in \tangent{x}$ onto a neighborhood $U$ of $x \in \mathcal{M}$. If, $\hat{U}$ is star-shaped (i.e., $\xi_x \in \hat{U}$ implies $a \; \xi_x \in \hat{U}$ for all $0 \leq a \leq 1$), then $U$ is called a normal neighborhood of $x$. We note that $\exponential{x}(t \; \xi_x) = \geodesic<l>{x}{\xi_x}(t)$ holds for every $t \in [0,1]$. The map
\begin{align*}
    \expOp \colon \; \tangent{} & \to \mathcal{M} \\
    (x,\xi_x) & \mapsto \exponential{x}(\xi_x).
\end{align*}
where $x$ is the foot of $\xi$, is differentiable, and $\exponential{x}(0_x) = x$ for all $x \in \mathcal{M}$. It can be shown that $\mathrm{D} \exponential{x}(0_x) [\xi] = \xi$ and consequently that $\expOp$ is a retraction (see \cite[Proposition~5.4.1]{AbsilMahonySepulchre:2008}). The exponential map generalizes the concept of moving “straight” in the direction of a tangent vector and is a natural way to update an iterate given a search direction in the tangent space. However, computing the exponential map is, in general, very complex. Therefore the concept of general retractions, \cref{Retraction}, is introduced to provide an alternative to the exponential map in the design of numerical algorithms that retains the key properties that ensure convergence results \cite[p.~102-103]{AbsilMahonySepulchre:2008}. \\
A vector field $\xi$ on a curve $\geodesicSymbol$ satisfying $\frac{\mathrm{D}}{\mathrm{d}t} \xi = \nabla_{\dot{\geodesicSymbol}} \xi = 0$ is called parallel. Given $a \in \mathbb{R}$ in the domain of $\geodesicSymbol$ and $\xi_{\geodesic<s>(a)} \in \tangent{\geodesic<s>(a)}$, there is a unique parallel vector field $\xi$ on $\geodesicSymbol$ such that $\xi (a) = \xi_{\geodesic<s>(a)}$. The operator $\parallelTransport{\geodesic<s>(a)}{\geodesic<s>(b)}$ sending $\xi (a)$ to $\xi (b)$ is called parallel transport along $\geodesicSymbol$. In other words, we have
\begin{equation*}
    \frac{\mathrm{D}}{\mathrm{d}t} \parallelTransport{\geodesic<s>(a)}{\geodesic<s>(t)}(\xi(a)) = 0.
\end{equation*}
We denote by $\parallelTransport{x}{y} \colon \; \tangent{x} \to \tangent{y}$ the parallel transport along the unique shortest geodesic $\geodesic<l>{x}{\xi_x} \colon [0,1] \to \mathcal{M}$, where $\geodesic<l>{x}{\xi_x}(0) = x$, $\geodesic<l>{x}{\xi_x}(1) = \exponential{x}(\xi_x) = y$ and $\dot{\geodesicSymbol}(0; x \, , \, \xi_x) = \xi_x$. The term $\parallelTransportDir{x}{\xi_x} = \parallelTransport{x}{y}$ is often used. \\
The parallel transport provides an idea of moving tangent vectors between tangent spaces. However, since it is based on the idea of the exponential map it is also often too expensive to use in praxis \cite[p.~9]{Huang:2013}. Much like the exponential map is a particular retraction, the parallel transport is a particular instance of a more general concept called vector transport which, roughly speaking, specifies how to transport a tangent vector $\xi_x$ from a point $x \in \mathcal{M}$ to a point $\retract{x}(\eta_x) \in \mathcal{M}$. Neither the vector transport nor the retraction are standard concepts of differential geometry. The general vector transport is nevertheless closely related to the classic concept of parallel transport \cite[p.~169]{AbsilMahonySepulchre:2008}.

\begin{definition}[{\cite[Definition~8.1.1]{AbsilMahonySepulchre:2008}}]\label{VectorTransport}
    A vector transport on a manifold $\mathcal{M}$ is a smooth map 
    \begin{align*}
        \vectorTransportSymbol \colon \; \tangent{} \oplus \tangent{} & \to \tangent{} \\
        (\eta_x, \xi_x) & \mapsto \vectorTransportDir{x}{\eta_x}(\xi_x)
    \end{align*}    
    satisfying the following properties for all $x \in \mathcal{M}$:
    \begin{enumerate}
        \item (Associated retraction) There exists a retraction $\retractionSymbol$, called the retraction associated with $\vectorTransportSymbol$, such that the following diagram commutes \begin{equation*}
        \begin{xy} \xymatrix{(\eta_x, \xi_x) \ar[r]^{\vectorTransportSymbol} \ar[d] & \vectorTransportDir{x}{\eta_x}(\xi_x) \ar[d]^\pi \\ \eta_x \ar[r]_{\retractionSymbol} & \pi (\vectorTransportDir{x}{\eta_x}(\xi_x)) } \end{xy} \end{equation*} where $\pi (\vectorTransportDir{x}{\eta_x}(\xi_x))$ denotes the foot of the tangent vector $\vectorTransportDir{x}{\eta_x}(\xi_x)$. \label{VectorTransport1}
        \item (Consistency) $\vectorTransportDir{x}{0_x}(\xi_x) = \xi_x$ for all $\xi_x \in \tangent{x}$; 
        \item (Linearity) $\vectorTransportDir{x}{\eta_x}(a \xi_x + b \zeta_x) = a \vectorTransportDir{x}{\eta_x}(\xi_x) + b \vectorTransportDir{x}{\eta_x}(\zeta_x)$.
    \end{enumerate}
\end{definition}

\cref{VectorTransport1} means that $\vectorTransportDir{x}{\eta_k}(\xi_x)$ is a tangent vector in $\tangent{\retract{x}(\eta_x)}$, where $\retractionSymbol$ is the retraction associated with $\vectorTransportSymbol$. The associated retraction of the parallel transport, $\parallelTransportSymbol$, is the exponential map, $\expOp$. When it exists,
$(\vectorTransportDir{x}{\eta_x})^{-1}(\xi_{\retract{x}(\eta_x)})$ belongs to $\tangent{x}$. Sometimes the term $\vectorTransport{x}{y} = \vectorTransportDir{x}{\xi_x}$ is used, where $y = \retract{x}(\xi_x)$ and $\retractionSymbol$ is the associated retraction. \\
A vector transport $\vectorTransportSymbol^S \colon \; \tangent{} \oplus \tangent{} \to \tangent{}$ with associated retraction $\retractionSymbol$ is called isometric if it satisfies for all $x \in \mathcal{M}$
\begin{equation}\label{IsometricVectorTransport}
    g_{\retract{x}(\eta_x)}(\vectorTransportDir{x}{\eta_x}(\xi_x)[S], \vectorTransportDir{x}{\eta_x}(\zeta_x)[S]) = g_x (\xi_x, \zeta_x)
\end{equation}
for all $\eta_x, \xi_x, \zeta_x \in \tangent{x}$ \cite[p.~10]{Huang:2013}. We use $\vectorTransportSymbol^S$ to denote an isometric vector transport. If the affine connection $\nabla$ on $\mathcal{M}$ is the Riemannian connection, \cref{RiemannianConnection}, then is the parallel transport $\parallelTransportSymbol$ isometric. \newpage
A vector transport by differentiated retraction, $\vectorTransportSymbol^{\retractionSymbol} \colon \; \tangent{} \oplus \tangent{} \to \tangent{}$, is a vector transport given by
\begin{equation*}
    \vectorTransportDir{x}{\eta_x}(\xi_x)[\retractionSymbol] = \mathrm{D} \; \retract{x}(\eta_x)[\xi_x]
\end{equation*}
i.e.,
\begin{equation}\label{DifferentiatedRetraction}
    \vectorTransportDir{x}{\eta_x}(\xi_x)[\retractionSymbol] = \frac{\mathrm{d}}{\mathrm{d}t} \retract{x}(\eta_x + t \; \xi_x) \; \vert_{t=0}
\end{equation}
where $\retractionSymbol$ is a retraction. We point out that in this case the chosen retraction, $\retractionSymbol$, in \cref{DifferentiatedRetraction} is the associated retraction of the resulting vector transport \cite[p.~172]{AbsilMahonySepulchre:2008}. We use $\vectorTransportSymbol^{\retractionSymbol}$ to denote a vector transport by differentiated retraction ($\retractionSymbol$ respectively). \\